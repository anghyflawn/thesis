\chapter{Introduction}
\label{cha:introduction}

This thesis presents an approach to phonological computation and representation which combines the tenets of substance\hyp free phonology, a framework which implies that phonological representation and computation are entirely agnostic of the physical realization of phonological units, with an explicit computational approach based on Optimality Theory. In order to explore the specifics of this framework, I undertake an extended comparison of the phonologies of two varieties of Brythonic Celtic.

The thesis explores a rather strong version of \emph{feature-based contrastivism}, an approach that rests on three important assumptions. First, it takes very seriously the idea that \emph{features} rather than \emph{segments} or \emph{inventories} are the first-class citizens of phonological computation. Second, it  includes the Contrastivist Hypothesis, which states that the phonological grammar of a given language operates precisely on the set of features that are allowed to implement lexical contrast. Third, the present approach embraces \emph{explicit modularity} and focuses very firmly on the division of labour between the different components of grammar in accounting for the sound pattern of a given language. In order to elaborate this approach, I explore a \emph{minimalist} framework, where phonological computation, as far as possible, does not involve elements of the grammar which are not warranted independently.

In order to demonstrate the merits of the substance\hyp free approach, I engage with the task of accounting for cross\hyp linguistic variation. While such variation has been a cornerstone of much recent work in theoretical phonology, here I take issue with several assumptions that are widespread in recent literature on the subject. In particular, I disagree strongly with the assumption that variation is solely produced by the phonological computation, with no contribution from representation. Instead, I advocate a model where input inventories built according to well-defined representational principles are filtered through the computational system to produce the attested inventories and patterns. Embracing this framework leads us to a rethink of the traditional rôle of factorial typology and the notion of \enquote{restrictiveness} that has been so prominent within work on Optimality Theory.

The present thesis contributes to an explicit theory of cross\hyp linguistic variation in substance\hyp free phonology by exploring the sound patterns of two closely related languages, namely the Welsh dialect of Pembrokeshire and the Breton dialect of Bothoa, both belonging to the Brythonic subgroup of the Celtic group of languages. In the chapters that follow I provide a comprehensive analysis of the phonology of these two languages which brings out the true similarities and differences in their systems.

As is to be expected, the phonological grammars of the languages demonstrate important differences. However, I also show that closer attention to \emph{phonological} representation brings out some aspects of cross\hyp linguistic variation that cannot be due to the computation alone, and which must be explained by other factors. This includes both the assignment of phonological features and consequent shape of phonological classes and, more importantly, the mapping between phonology and phonetics. Specifically, I show that segments which are \enquote{pronounced the same} in the two languages can have very different phonological representation, which is not a very new insight. More importantly, I show that segments which differ phonetically in ways that have been claimed to correspond to different phonological representation in fact have very similar phonological structure and behaviour: among other proposals, I advocate a revision of the set of assumptions known as \enquote{laryngeal realism} which breaks the link between the phonetic realization of laryngeal contrasts and their phonological structure.

These results have the very important implication that phonetics does not determine phonological representation, which, in turn, means that any study of cross\hyp linguistic variation cannot \emph{prima facie} rely on the assumption that we can reliably extract phonological patterns from transcribed data. Instead, cross\hyp linguistic comparison must rely on in\hyp depth phonological analyses of the relevant languages. In this thesis I emphasize the following analytic techniques to achieve this goal:

\begin{itemize}
\item \emph{Explicit modularity.} Phonology is a separate module of grammar, with non-trivial interfaces to other distinct modules such as morphosyntax and phonetics. Phonology operates with its own set of primitives and computational operations, which are not available to the other modules and have to be translated in a non\hyp trivial manner at the interfaces;
\item A practical consequence of this principle for the analyst is what I call the \emph{presumption of guilt}. In a theory where language-particular manipulation of sound patterns (broadly understood) can happen at several points in the derivation, the fact that some phenomenon can be understood as, say, an alternation, does not automatically mean that it falls into the purview of phonological theory. On the contrary, it has to satisfy several well-defined criteria to be classified as a phonological process or a matter of the phonetics--phonology interface, or assigned some other function;
\item \emph{Categoricity.} I subscribe to the view that the phonological component deals in categorical operations on discrete elements. However, I reject the assumption that categoricity \emph{defines} what phonology is: categorical behaviour can be produced as an epiphenomenon of non\hyp phonological operations.
\end{itemize}


On the computational side, this thesis uses Optimality Theory, as it has a number of well-documented advantages. However, the representational proposals made in the thesis can hopefully be useful independently of one's computational model. Moreover, the rejection of substance-based (and other straightforwardly \enquote{functional}) factors in favour of a simpler computational system making generous use of constraint schemata means that the predictions made here may not be immediately comparable to the more specific predictions of a more orthodox OT analysis. More generally, I suggest that the predictions of the theory of phonological computation, \ie the restrictions that it puts on the set of possible languages, are of an architectural nature: the theory of phonology can predict the type of operations on phonological symbols that should be (im)possible, but it is entirely agnostic with respect to the substantive effects of these operations.

One particular consequence of this approach is the rejection of substantive factors in the formulation of OT constraints. For instance, in this thesis I make liberal use of a constraint schema that requires certain phonological structures to be accompanied by other structures in the surface representation. Such constraints are far from unknown in the literature; however, their status is quite ambiguous. They are often rejected under the guise of \enquote{positive markedness constraints}; if they are admitted to the constraint set \textsc{Con}, this is usually done mostly to express certain (functionally grounded) asymmetries between \enquote{strong} and \enquote{weak} positions. Since such considerations are irrelevant in substance\hyp free phonology, I freely admit such augmentation constraints, and argue that their undesirable properties in terms of factorial typology (as traditionally understood) do not outweigh their analytic advantages.

A second major computational point concerns the interactions between phonology and morphology and associated problems such as opacity. In this thesis I use a stratal model of Optimality Theory, which inherits many of the assumptions of rule\hyp based Lexical Phonology, in particular the distinction between three levels of phonological computation (stem\hyp level, word\hyp level, and postlexical). I argue that this approach has a number of important advantages over competing approaches (such as lexically indexed constraints, cophonologies, or serial OT formalisms) both with regard to the data at hand and in architectural terms, especially where modularity is concerned. While the present thesis certainly cannot resolve this important issue, it is to be hoped that it will add to the growing body of evidence brought to bear on this debate.

The thesis is organized as follows. In \cref{cha:subst-free-phon} I lay out the conceptual underpinnings of substance\hyp free phonology, which, in the present framework, rests on the assumption of a \emph{modular} architecture of grammar and consequent \emph{autonomy of phonology}. \Cref{cha:repr-assumpt} discusses the representational framework used in this thesis. Specifically, I present a version of the Parallel Structures Model of feature geometry and show how it can be reconciled with approaches based on a contrastive hierarchy of distinctive features. In \cref{cha:comp-assumpt} I lay out important computational concerns, in particular aspects related to computational complexity. I also present technical discussion of some constraints that will be important for the analyses and the basics of the stratal approach. Finally, in \cref{cha:categ-contr-mark} I discuss three notions that have commonly been taken to be very important to defining \enquote{what phonology is}: categoricity, the rôle of contrast, and the nature of phonological markedness.

\Cref{part:analysis} contains the body of the dissertation, \ie the two empirical studies which build on the theoretical foundation. \Cref{cha:bryth-lang} presents a brief overview of the Brythonic Celtic languages and some relevant literature. The phonology of Pembrokeshire Welsh is the subject of \cref{cha:pembrokeshire-welsh}, while \cref{cha:bothoa-breton} contains a description and analysis of the Breton dialect of Bothoa. Some discussion of the repercussions of these analyses and of alternative approaches to some of the data is found in \cref{cha:disc-altern-analys}. \Cref{cha:concl-repr-vari} concludes and provides some avenues for further enquiry.