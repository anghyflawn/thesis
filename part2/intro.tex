\chapter{The Brythonic languages: an overview}
\label{cha:bryth-lang}

In this brief chapter I provide a very short overview of the Brythonic languages, giving the minimum necessary information regarding their common historical background, dialectal variation, and status of description. I also provide a brief discussion of the status of initial consonant mutations, which have been extremely prominent in the discussion of the phonology of the Brythonic languages in the theoretical literature.

The Brythonic languages are a subgroup of the Celtic group, comprising Welsh, Breton, and Cornish. Welsh is spoken in Wales, mainly in the rural north and west, but also in the industrial valleys of the southeast and in some of the most important cities. Breton is spoken in Lower Brittany, \ie in the western part of the historic duchy. Native speakers are mostly spread throughout rural Brittany, and the language is severely endangered. Cornish, previously spoken in Cornwall, became extinct in the 18th or 19th century, although efforts are underway to revitalize the language. I will not consider data from Cornish in any detail in this thesis.

\section{A historical overview}
\label{sec:historical-overview}

An important motivation for the comparison undertaken in this thesis is the relative similarity of the phonological patterns (\ie inventories and alternations) of the languages under consideration, which brings the cross\hyp linguistic variation into greater relief than would be possible if the comparison concerned two widely divergent varieties. This similarity is largely due to common origin. In order to set the scene for later discussions, in this section I provide a brief overview of some of the most important features of the historical phonology of Brythonic Celtic which provide a common backdrop for some alternations to be discussed in detail in \cref{cha:pembrokeshire-welsh,cha:bothoa-breton}.

\subsection{The obstruent system}
\label{sec:obstruent-system}

The reconstruction of the obstruent system of Proto\hyp Brythonic remains a contentious issue. Traditional scholarship \citep[\egm][]{wg-mj,lewis37:_celtic} recognized an independent four\hyp way distinction between voiced and voiceless singleton and geminate consonants in word\hyp medial position. It must be noted, however, that voiced geminates were rare, and became singletons at a relatively early stage, as in *\emph{ad-bero-}~$\rightarrow$ *\emph{abbero-}~$\rightarrow$ Welsh \emph{aber} `estuary'; I will return to this issue below in \cref{sec:concl-typol-impl}. This reconstruction was rejected by \citet{lheb,jackson60}, who emphasized a distinction between \enquote{fortis} and \enquote{lenis} obstruents in all positions \citep[\cf also][]{oftedal-lenition}, and proposed that laryngeal distinctions in later languages, usually described in terms of voicing, were largely derivative from this basically quantitative contrast. This reconsideration was motivated by the theoretical work of \citet{mart55} and by the empirical findings of \citet{falchun38:_recher,falchun}, who also emphasized the primacy of the quantity contrast in Breton phonology. Data very similar to those that played an important rôle in these developments are discussed below in \cref{sec:leonais-breton}.

The principal distinction between the two accounts in phonological terms lies in the nature of markedness relations. The traditional account assumes that both Proto\hyp Brythonic and the modern languages contrast two classes of obstruents, where the \enquote{voiced} ones are more marked. Evidence for this could be found in the avoidance of voiced geminates (which can be formalized if we assume a markedness constraint mentioning [(+)voice]) and the existence of final devoicing in Breton (and possibly Cornish)\dash a classic hallmark of the markedness of [voice] \citep[\cfm][]{harris09:_why_final_obstr_devoic_is_weaken}. The Jacksonian account emphasizes the quantitative aspect of the modern facts, and thus treats \enquote{fortis} obstruents as more marked. Possible corroboration for this point of view can be found in the fact that in all dialects of Welsh, and some dialects of Breton (Le Bourg Blanc \dash \citealt{falchun}; Île de Groix\dash \citealt{Ter70}) the descendants of the fortis stops (\ie the traditional \enquote{voiceless}) are realized with aspiration. Further evidence for the marked status of \enquote{voiceless}\fshyp \enquote{fortis} obstruents in Brythonic Celtic may be found in the process of \enquote{new lenition}\dash across\hyp the\hyp board voicing of fricatives in some Breton varieties and in Cornish \citep{histbreton,tristram95:_zaoz_zomer,hewitt99:_remar_breton,chaudhri}. This process appears similar to Southern English Fricative Voicing, used by \citet{honeybone05} to argue for a similar markedness situation in English.

The issue has not been settled yet: \citeauthor{lheb}'s approach was criticized by \citet{harvey84,thomasa,schrijver}, while \citet{koch-prosody,koch89,koch90,isaac,isaac08:_bryth} proposed accounts that uphold some version of a system where the \enquote{voiceless} obstruents are more marked phonologically throughout the history of the Brythonic languages. In turn, \citet{sims-williams08,sims-williams10:_welsh} engages with the arguments of \citet{isaac,isaac08:_bryth}.

Resolving this historical issue is far beyond the scope of the present thesis; what is important is that the question of which laryngeal state corresponds to greater phonological markedness in Brythonic Celtic is non\hyp trivial, and deserves serious consideration. In the chapters that follow I will argue that the \emph{modern} languages (at least the two varieties that I consider) treat \enquote{voiceless}, or \enquote{fortis}, obstruents as more marked phonologically. Any application of these results to the state of affairs in Brythonic Celtic, however, must await future work.

\subsection{The Brythonic quantity system and the accent shift}
\label{sec:bryth-quant-syst}

There are two important events in Brythonic historical phonology that have exerted a significant influence on the sound patterns of the modern languages: the establishment of the so\hyp called \enquote{new quantity system} and the \enquote{accent shift}. The patterns created by these changes play a prominent rôle in the present thesis.

Proto\hyp Celtic, like other ancient Indo\hyp European languages, did not put significant restrictions on the distribution of long vowels: they could be found both in stressed and unstressed syllables, irrespective of the existence of a syllable coda, and length in general tended to function independently of stress. A series of changes in the Proto\hyp Brythonic era established a new system of vowels, which differed from the Proto\hyp Celtic one in two important respects (for overviews, \cf \citealt{lheb,mccone,schrijver95:_studies_britis_celtic,schrijver11:_old_britis}). First, the old quantitative contrasts were largely transformed into qualitative ones: for instance, while proto\hyp Celtic \emph{*ī} became Proto\hyp Brythonic *\ipa{\emph{i(ː)}}, proto\hyp Celtic \emph{*i} was reflected in Proto\hyp Brythonic as a sound sometimes written as \emph{\ipa{*ɪ}} (Breton and Cornish \emph{e}, Welsh \ipa{\emph{ə}}, \ipa{\emph{ɨ(ː)}}, \ipa{\emph{i(ː)}} depending on dialect and position). Second, the new qualitative contrasts were now cross\hyp cut by a new quantity distinction in vowels. The quantity differences arose in stressed syllables: under (main) stress, vowels were long before singleton consonants but not before geminates or consonant sequences (I use the acute accent to indicate stress): Early Proto\hyp Brythonic \emph{*tátos}~$\rightarrow$ Late Proto-Brythonic \emph{\ipa{táːtoh}}~$\rightarrow$ Welsh \ipa{[ˈtaːd]}, Breton \ipa{[ˈtaːt]} `father' but Early Proto\hyp Brythonic \emph{*trumbos}~$\rightarrow$ Welsh \ipa{[ˈtrum]} `heavy'. At the same time these developments led to a situation where the length contrast was severely weakened or altogether obliterated in pretonic syllables.

Following apocope (\ie the deletion of final syllables), stress, which in Proto\hyp Brythonic used to fall on the penultimate syllable, was now final, and thus it was the final syllable that was the locus of the restrictions due to the \enquote{new quantity system}. Following certain development in the consonant system, such as the voicing of intervocalic stops and the change of voiceless geminate stops \ipa{\emph{*pː~*tː~*kː}} into fricatives \ipa{\emph{f~θ~χ}}, these restrictions came to be seen as a dependence between vowel length and consonant quality.

This state of affairs appears to persist in the peripheral Vannetais dialect(s) of Breton, where stress remains final to this day \citep[\egm][]{Ter70,mckenna,le00:_le_malguen,le08:_etude,cheveau07:_approc_grand_lorien}; however, it has also been argued that Vannetais final stress is due to later influence from French. Most other Brythonic dialects underwent a retraction of stress to the penultimate syllable. The date of this shift remains controversial \citep{lheb,watkins72:_old_welsh,watkins76:_cyfnew_gymraeg}, but for our purposes it is important to recognize two important consequences.

First, the accent shift led to a reorganization of the vowel quantity system. It had no effect on vowel length in monosyllables, where its application was vacuous. However, in polysyllables length was overwhelmingly found in stressed syllables. A retraction of stress to a preceding syllable could create a mismatch between length and stress. The uniform response was a shortening of long vowels in word\hyp final, newly unstressed syllables: long vowels in final syllables are indeed found in the modern languages, but they always stem from later developments (collectively known as contractions; see \cref{sec:synaresis} for more details).

Due to the earlier weakening of the vowel length contrast in unstressed position, the newly stressed vowels were usually short. This situation persists in Northern Welsh, where a distinction in vocalic quantity is only found in word\hyp final stressed syllables (\ie overwhelmingly in monosyllabic words): vowels in stressed penultimate syllables are always short, irrespective of the properties of the following consonant (see \cref{sec:vowel-length} for examples). However, in South Welsh, in Cornish, and in Breton dialects which had undergone the stress shift, some newly stressed vowels are reflected, at least historically, with \enquote{half\hyp length}. The distribution of these \enquote{half\hyp long} vowels closely mirrors the distribution of long vowels in word\hyp final stressed syllables, although  as we shall see in \cref{sec:vowel-length} the two positions do not always show identical behaviour, at least in Welsh. The synchronic facts related to these developments will play a major rôle in this thesis.

Another, less far\hyp reaching consequence of the accent shift is seen in Welsh, where (unstressed) word\hyp final syllables often exhibit a high degree of phonetic prominence, due to a significant rise in pitch on the post\hyp tonic syllable that is seen in certain prosodic constructions. This rising pitch is commonly agreed to be the remnant of the erstwhile word\hyp final stress \citep{watkins76:_cyfnew_gymraeg}, and it has also played a rôle in formal phonological analysis \citep{bosch96:_promin}. I take up related issues in \cref{sec:vowel-mutation}.

\section{Breton}
\label{sec:breton}

Breton is spoken in the western part of the peninsula, traditionally called Lower Brittany (French \emph{Basse-Bretagne}; Breton \emph{Breizh-Izel}), as opposed to the Romance-speaking Upper Brittany.\footnote{Apart from varieties of Standard French, the local Gallo-Romance variety, called \emph{gallo}, is also spoken in Upper Brittany.}

\subsection{Dialects}
\label{sec:dialects}

Traditionally, Breton is divided into four major dialect groups, on the basis of the old diocesan borders. These are as follows:

\begin{itemize}
\item Cornouaillais (\emph{Kerneveg}), the dialect of Cornouaille, the biggest of the dioceses covering the south-west corner of Lower Brittany and most of its inland region. The traditional centre of the diocese is the city of Quimper (\emph{Kemper}); today it covers the southern part of the \emph{département} of Finistère (\emph{Penn-ar-Bed}), and also includes some regions in the south of the \emph{département} of Côtes d'Armor (\emph{Aodoù-an-Arvor}) and in the north-west of Morbihan (\emph{Mor-Bihan});
\item Léonais (\emph{Leoneg}) is the dialect of Léon, the diocese in the north-west of Brittany centred around Saint-Pol-de-Léon (\emph{Kastell-Paol}); today the northern part of Finistère;
\item Trégorrois (\emph{Tregereg}), in the north-east of Brittany, and in today's \emph{département} of Côtes-d'Armor. This also includes the dialects of Goëlo, a small area in the extreme north-east of the Breton-speaking region, which belongs to the otherwise entirely Romance-speaking traditional diocese of Saint Brieuc (\emph{Sant-Brieg});
\item Vannetais (\emph{Gwenedeg}), spoken in the south-eastern part of Brittany, in the traditional diocese of Vannes (\emph{Gwened}).
\end{itemize}

Traditionally, it is often assumed that the first three dialects are relatively homogeneous, and they are sometimes referred to together as a single dialect grouping called KLT, which is opposed to the Vannetais dialect. The basis for this division would seem to be partly philological and partly sociolinguistic. From the philological standpoint, Vannetais presents a number of striking differences with respect to the other dialects. Most prominently, it has final stress where other Brythonic varieties have undergone retraction (\cref{sec:bryth-quant-syst}) (although it is not universally agreed whether this is a shared innovation or a retention); on the other hand, Vannetais dialects share the sound change of proto-Brythonic *\ipa{θ} to \ipa{[h]}. From a sociolinguistic perspective, Vannetais has had a literary tradition separate from the other Breton dialects \citep[\cfm][]{guillevic02}.

Nevertheless, these distinctions are not as clear-cut as the above picture suggests. As discussed by \citet[§§23--27]{histbreton}, it may be more accurate to describe Léonais and Vannetais as genuine dialect groupings (though with significant internal diversity), whereas Trégorrois and Cornouaillais are best described as a more or less homogeneous single \enquote{central} dialect. Trégorrois is said to have undergone significant influence from Upper (\ie Eastern) Léonais, whereas Cornouaillais, according to \citet{histbreton}, is an area characterized merely by not having some features distinguishing Léonais on the one hand and Vannetais on the other; \textcite{hewitt73:_moder_liter_breton} aptly calls it \enquote{a dialect by default}.

The situation of written Breton is quite precarious, since it is very little used by native speakers (\citeauthor{hewitt73:_moder_liter_breton} \cite*{hewitt73:_moder_liter_breton}) and in addition suffers from the existence of competing orthographic standards \citep{wmffre07_1,wmffre07_2}; for a general overview, see e.\,g. \citet{hewitt73:_moder_liter_breton,JONES01101995}. The written standard(s) are, for historical and political reasons, in important respects based on the dialect of Léon; some implications of this are discussed by \citet{hewitt73:_moder_liter_breton,wmffre07_1}, who are in general quite critical of the standard language, finding it too far removed from the Breton of native speakers (as opposed to the \emph{néo-bretonnants} who have learned it in a formal setting as a second language). Especially in terms of representing the sound system, the prevalent orthography (the so-called \emph{orthographie unifiée}) in some respects sacrifices consistency in the name of providing a single norm for all dialects \citep{hewitt73:_moder_liter_breton,wmffre07_1,wmffre07_2,madeg10:_trait_nord_ouest}. In this thesis I will sidestep these issues, and, where I give the written form, I will follow the relevant source, even if it may differ from the dialect form at hand in some phonological or morphological details.\footnote{For Breton orthographic forms, I have used the dictionaries by \citet{hemon05} and \citet{cornillet06}. I have also used \citet{favereau97:_geriad_diction}, which is a descriptive work incorporating dialect forms, often containing forms not shown in the standard-oriented dictionaries but attested in the variety at hand; crucially, however, different editions of that dictionary use different orthographies.}

\subsection{Sources}
\label{sec:sources}

In this section I provide a short overview of some sources that contain treatments of the sound systems of various Breton dialects.

\subsubsection{General descriptions}
\label{sec:general-descriptions}

Most existing general overviews of Breton focus on the standard language, with only occasional and unsystematic remarks regarding the living dialects. This is particularly true of reference grammars, of which \citet{hemon40:_gramm}, \citet{yezhadur}, \citet{trepos66:_gramm},  and \citet{press} are  perhaps the most comprehensive. An exception is the grammar of \citet{favereau01}, which often presents a pandialectal, descriptive perspective (albeit with little reference to phonetics and phonology). Shorter overviews of varying depth are provided by \citet{breton-celtl,press-lincom,ternes93:_breton,ternes11:_neubr}.

With respect to dialectology, a very important source is the \textfrench{\emph{Atlas linguistique de la Basse-Bretagne}, or ALBB} \citep{le63:_atlas_basse_bretag}, based on data gathered in the period between 1911 and 1920.\footnote{These dates are significant; before World War I, much of Lower Brittany remained primarily Breton-speaking, whereas wartime service provided the impetus for a very large proportion of the population to learn French \citep{broudic95:_la_regim}.} It has served as a source of primary data for much subsequent work. A newer dialectological atlas is \citet{le01:_nouvel_basse_bretag}.

Two other important works must be mentioned here. \citet{histbreton} presents a comprehensive historical phonology of Breton. It is very useful not only in diachronic terms, but also as an important compendium of data that are otherwise scattered in disparate and often obscure sources. It should be noted, however, that there were relatively few comprehensive dialect descriptions available when this work was published, and this was especially true of Cornouaillais varieties (the biggest dialect area), a situation that has improved since then. Nevertheless, it remains a very important source.

A comprehensive overview of the dialect situation (with a diachronic outlook) was presented in several versions by François Falc'hun, presented in several publications culminating in \citet{falchun81:_persp}. He uses ALBB data to argue for a particular version of the history of Breton, making important contributions to systematic dialectology in the process.

For earlier stages of the language, the most important sources remain \citet{fleuriot64:_le} for Old Breton, \citet{LlydawegCanol} for Middle Breton and \citet{hemon75:_breton} for a historical perspective on morphosyntax; see also \citet{schrijver11:_middl_early_moder_breton,schrijver11:_old_britis} for shorter summaries.

\subsubsection{Dialect descriptions}
\label{sec:dialect-descriptions}

Coverage of the Breton-speaking area by systematic phonetic and/or phonological descriptions is quite uneven, as the following list (which is, however, far from exhaustive) will demonstrate.

\begin{itemize}
\item For Léonais, \citet{st-pol-de-leon} (originally published in 1922) is a study of the dialect of Saint-Pol-de-Léon, the original centre of the diocese. A milestone in Breton phonology is the study by \citet{falchun}, who concentrated on the contrast between \emph{fortes} and \emph{lenes} which has a played a central rôle in much diachronic and synchronic literature on Breton (see \cref{sec:leonais-breton}). \citet{falchun} focused on his native dialect of Le Bourg Blanc (\emph{Ar Vourc'h-Wenn}). Finally, \citet{carlyle88:_breton} presents a generative study of various aspects of the dialect of Lanhouarneau.
\item For Trégorrois, apart from the early \citet{botsorhel} and \citet{le08:_gramm_treguier}, two important sources are \citet{plougrescant} and \citet{le78:_le_ploug}, which both treat the dialect of the peninsula of Plougrescant, close to the major town of Lannion. \citet{plougrescant} presents a relatively short descriptive study, without a consistent phonological approach. On the other hand, \citet{le78:_le_ploug}, drawing on his knowledge as a native speaker and on extensive fieldwork, presents a comprehensive account of the phonology, morphology and lexicon of this variety of Breton, but he pays relatively little attention to phonetics and uses a structuralist phonemic notation which may or may not gloss over some phonetic details. Some data can also be gleaned from the short grammatical overviews in \posscite{le12:_le_ploug} dictionary. Other, more cursory descriptions are found in \citet{sommerfelt62:_notes_dourd_plouez_finis} (Plouézoc'h) and \citet{dressler73:_alleg_lentor} (Buhulien);
\item The Vannetais area is relatively well served by comprehensive descriptions: these include \citet{Ter70} for Île de Groix, \citet{mckenna} for Guéméné\hyp sur\hyp Scorff, \citet{le00:_le_malguen,le08:_etude} for Malguénac, \citet{cheveau,cheveau07:_approc_grand_lorien} for Grand\hyp Lorient. Shorter descriptions include \citet{thibault14:_notes_cleguer_morbih} for Cléguérec and \citet{plouharnel} for Plouharnel;
\item There are also several major studies of Cornouaillais, although they are often concerned with \enquote{peripheral} or \enquote{transitional} varieties. An important work is the phonetic study by \citet{Bot82} for Argol in the Crozon peninsula; other works on \enquote{core} Cornouaillais dialects include \citet{dressler80:_etudes,sinou99:_le_lechiag,sinou00:_brezh_lechiag} for the extreme south-west, \citet{denez77:_descr_douar} for Douarnenez, and \citet{timm84:_carhais_breton,favereau84:_langue_poull,wmffre99:_centr_breton} for the environs of Carhaix. For transitional zones, we find \citet{ploneis} for Berrien on the border with Léon, \citet{humphreys,humphreys95:_phonol_bothoa_saint_nicol_pelem} for Bothoa in the far north-east, and \citet{evenou87:_studi_lanij_faouet_kernev}, with a short French summary in \citet{evenou89:_descr_lanven}, for Lanvénégen on the border with Vannetais.
\end{itemize}


In addition to this selection of sources, information on the phonetic and phonological make-up of the relevant dialects can sometimes be gleaned from the numerous publications describing the dialectal lexicon (though these often use the orthography), and from sound materials published by organizations such as \emph{Dastum} dedicated to preserving the sound heritage of Brittany.

\subsubsection{Theoretical studies}
\label{sec:theoretical-studies}

Breton phonology has not been the subject of great attention from theoretical phonologists working in the generative tradition. Moreover, the existing body of literature is heavily biased towards issues related to initial consonant mutation. Initial consonant mutations in Breton are often considered together with those of other Celtic languages, in particular the closely related Welsh; some examples here are \citet{willis,Pya97,green2003,greenbook}. \citet{stump-mixed,stump-nonlocal} deals with both phonological and morphological aspects of mutation, while prosodic conditioning of some Breton mutations is considered by \citet{pyatt}. \citet{wolf2005,wolf-forautosegs} proposes an account of some phonological aspects of the mutation system.

A very different approach is undertaken by \citet{dressler73:_alleg_lentor,dressler80:_etudes}, who consider Breton data in the context of speech rate-related phonological subsystems.

Issues related to sandhi voicing and devoicing in a Vannetais dialect are treated in detail by \citet{kramer-breton} and \citet{hall09:_laryn_breton}. A relatively complete study of the segmental and prosodic system of Léonais Breton is undertaken by \citet{carlyle88:_breton}.

\section{Welsh}
\label{sec:welsh}

Welsh is spoken in Wales, with the strongest Welsh-speaking areas found in the sparsely populated and mostly rural north and west (Anglesey\fshyp Ynys Môn, Gwynedd, Ceredigion, Powys, Carmarthenshire\fshyp Sir Gaerfyrddin, Pembrokeshire\fshyp Sir Benfro), but also in the south\hyp east, especially in the industrial valleys.

\subsection{Dialects}
\label{sec:dialects-1}

It is commonly acknowledged that the most important dialect boundary in Wales is that between the north and the south, which is defined by a number of isoglosses (mainly phonological and lexical) of which the northernmost go slightly to the north of the mouth of the River Dyfi (Dovey) and the others go progressively to the south, with much of Mid Wales (mostly the county of Ceredigion) as a transitional zone.

Within these groups, the north\hyp west (northern Gwynedd and Anglesey\fshyp Môn) and the south\hyp east (Glamorgan) also show significant important differences vis-à-vis north\hyp eastern and south\hyp western dialects respectively. I will not concentrate on the dialectal diversity too much, as descriptions are readily available in the literature, \cf in particular \citet{thomas73:_wales,welshphonotactics,awbery09:_welsh,thomas89:_cymraeg_cymra_cymre}.

\subsection{Sources}
\label{sec:sources-welsh}

The volume of both descriptive and theoretical literature treating Welsh is rather large, especially if one takes into account the numerous graduate theses of the University of Wales, so I will not attempt a similarly full review of the literature, quoting only those sources which I have been able to use.

\subsubsection{General descriptions}
\label{sec:general-descriptions-welsh}

Welsh has a strong literary tradition and a large number of overview grammars concentrating both on the traditional written language \citep[\egm][]{wg-mj,welshsyntax,thorne93:_compr_welsh_gramm,gyg} and on the emerging vernacular standard \citep{king}. Systematic overviews of the language are also found in reference works, such as \citet{awbery09:_welsh}. Individual subsystems of the language are also covered rather well, see \citet{ball01:_welsh_phonet} for phonetics, \citet{welshphon} for phonetics and phonology, \citet{miw} for consonant mutations, and \citet{syntaxofwelsh} for syntax.

Old Welsh is described in detail by \citet{falileyev07:_le} and, more concisely, by \citet{schrijver11:_middl_early_moder_breton}. The standard work for Middle Welsh remains \citet{simonevans}, although \citet{wg-mj} is also useful. A more recent treatment, also covering Early Modern Welsh, is provided by \citet{schumacher11:_mittel_fruhn}.

\subsubsection{Dialect descriptions}
\label{sec:dialects-2}

Welsh dialects have been studied rather extensively, although the majority of existing descriptions are still somewhat cursory with regard to phonetics and phonology. They are also often either historically oriented or structuralist in approach, meaning that they contain a great deal of pre\hyp analysis.

Many of these descriptions exist as graduate theses of the University of Wales. \citet{thomas-sewales} presents an overview of south\hyp eastern dialects based on the results of such theses. Published descriptions of dialects include the following:

\begin{itemize}
\item For South Welsh
  \begin{itemize}
  \item For south-eastern dialects, \ie Glamorgan and the valleys, \citet{thomas64:_rhai_nantg,thomas93:_tafod_nantg} is a comprehensive description of the dialect of Nantgarw. There is also \citet{thomas60:_ffonem_dyffr_wysg,thomas61:_ffonem_dyffr_wysg} for the dialect of the Usk Valley.
  \item For south-western dialects, two important works are \citet{awbery86:_pembr_welsh} for Pembrokeshire Welsh and \citet{thorne76:_astud_carnw_sir_gaerf} for the Llanelli area (south-east Carmarthenshire), in addition to \citet{llansamlet} for Llansamlet (a suburb of Swansea). The main features of the south\hyp west dialects are presented in the popular \citet{jones92:_dyfed}.
  \end{itemize}
\item For Mid Welsh (\ie the counties of Ceredigion and Powys), there exist a series of studies by \citet{pilch57:_lauts,pilch-nasal,pilch75:_advan_welsh} for Bow Street near Aberystwyth in Ceredigion, \citet{sommerfelt} for north\hyp west Powys, and \citet{brycheiniog} for south-east Powys, \ie the former county of Breconshire. In addition, \citet{wmffre03:_languag_wales} presents an overview of some major features of dialects in Ceredigion on the basis of placenames (but also with reference to other descriptions of the relevant varieties).
\item For North Welsh, a major source is the lexicographical description of the Bangor dialect (with transcriptions) by \citet{fynes-clinton}. A relatively complete description of the phonology of the Vale of Alun dialect is given by \citet{thomas66:_system_welsh}.
\end{itemize}

Finally, \citet{thomas73:_wales} presents a geographical study of Welsh dialects, and \citet{thomas00:_welsh} is a publication of the materials of the Welsh dialect survey, containing much interesting primary data.

Relevant theoretical publications, which are quite numerous, are cited and discussed as appropriate below in \cref{cha:pembrokeshire-welsh,cha:disc-altern-analys}.


\section{The status of initial consonant mutations}
\label{sec:stat-init-cons}

In the discussion that follows I will occasionally use some evidence from the sound patterns of initial consonant mutations (for a recent overview of the issues, see \citealp{hannahs11:_celtic}). In the vast majority of cases, I will assume that initial consonant mutations are the product of the prefixation of a floating feature (see \citealp{lieber,lieberbook,swingle,wolf2005,wolf-forautosegs}; \cf also \citealp{hamp,roberts}). However, it is not my intent in this thesis to provide a full theory of initial consonant mutations.

The issue here is the lexical and morphosyntactic affiliation of the putative autosegments. Even when the phonological rationale can be accounted for, it is not always entirely clear how the relevant autosegment is inserted.

Traditionally in generative phonology, Celtic initial mutations have been accounted for in terms of morphosyntactically triggered, often extrinsically ordered rules \citep[\egm][]{rogers,kibre}. Morphosyntactic triggering is problematic for reasons of modularity, while extrinsic ordering is generally problematic; it has been most commonly deployed to account for chain shifts. Autosegmental accounts have replaced morphosyntactic triggering conditions by lexical insertion, by treating the autosegments either as part of the lexical representation of mutation triggers or as exponents of certain morphological categories \citep[\cfm][]{swingle,wolf-forautosegs}.

However, this approach is not sufficient to fully derive the data, since in many cases the mutation environments do not seem to line up neatly with morphosyntactic conditioning: for instance, it is common to treat certain gender\hyp related mutations in Welsh as markers of gender agreement \citep[\egm][]{kibre}, although the distribution of mutation is not identical to the distribution of agreement markers (see \citealp{leipzig} for more discussion).

In addition, in some contexts initial mutation is extremely difficult to motivate using an autosegment. The prime example here is the \enquote{XP\hyp trigger hypothesis} used to explain the triggering of certain mutations in Welsh \citep[\egm][]{harlow,xptrig,borsleymut,tallermanforth,syntaxofwelsh}, formulated in a number of different ways. \citet[p.~247]{syntaxofwelsh}, following \citet{borsleymut}, settle on the following formulation:

\begin{displayquote}
  A complement bears S[oft] M[utation] if it is immediately preceded by a phrasal sister.
\end{displayquote}

Formalizing this type of mutation would require inserting an autosegment between two adjacent phrasal sisters, which is very difficult to reconcile with autosegments being treated as morphemes expressing morphological categories.\footnote{\citet{lieberbook}, following \citet{rice-cowper,conteh-cowper-rice}, proposes to treat mutation in Mende as being due to an autosegmental \enquote{clitic} also inserted in very general morphosyntactic conditions having to do with adjacency and c\hyp command; see, however, \citet{vydrine,iosad-toulouse} for arguments that this is a misanalysis.} Similarly, \citet{tallerman} argues that some instances of mutation in Welsh are best seen as marking deviations from the expected word order, which is again very difficult to square with an autosegmental triggering mechanism.

Finally, the existence of chain shifts remains problematic. Although devices have been proposed in the literature to treat chain shifts in parallel OT frameworks \citep{kirchner-lc,wolf2005,wolf-forautosegs}, their status remains controversial. A related aspect is that some mutations (including chain\hyp shifting ones) apparently involve deletion of segments, which is notoriously difficult for additive models of morphology \citep[\cfm][]{bye09:_exten_expon_and_non_concat_morph}, and which also involves some morphosyntactic difficulties \citep{welsharticle}.

All these difficulties have led some scholars to argue that processes such as mutation are properly outside the domain of phonological computation \citep{stewart,green2003,greenbook,bye-allomorphy,leipzig}, which would thus invalidate mutation evidence as a tool to gain insights into phonological patterns. At the same time at least some processes have been shown to interact with phonological structure such as prosody \citep{pyatt}, which would seem to put them back into the phonological component.

Since a major point of the present thesis is disentangling the affiliation of sound patterns to different components of grammar, I will not make any \emph{a priori} pronouncements on whether mutation is general is a phonological process. It must be noted that the largest number of theoretical challenges has been connected with just one type of mutation, called \enquote{soft mutation} in Welsh (\cpageref{sec:soft-mutation}) and \enquote{lenition} in Breton (\cref{sec:lenition}), whereas other mutations submit relatively easily to an autosegmental treatment.

In the chapters that follow I will assume that an autosegmental analysis based on the insertion of floating features as exponents of morphological categories is possible (see in particular \cref{sec:spirantization}) for most cases. However, I will not go into detail regarding the precise mechanisms of segment deletion, and in particular of chain shifts. For the sake of the argument, I will assume that the grammar uses input subcategorization \citep{paster06:_phonol,bye-allomorphy,yu07} to insert the \enquote{correct} autosegment, and thus that chain shifting is not derived by the phonology (for a slightly different approach, \cf \citealp{wolf2007}); for more discussion, see \cref{sec:analysis-6,sec:analysis-4} below. It is clear that more work is needed to ascertain the morphosyntactic status of the autosegmental devices I propose to derive initial mutation. I leave these issues for further research.
