\chapter[Representation and the sources of variation]{Conclusion: representation and the sources of variation in substance\hyp free phonology}
\label{cha:concl-repr-vari}

In this thesis I have presented a comprehensive analysis of the phonology of two closely related varieties of Brythonic Celtic. Although the systems, \ie the phonological inventories and the patterns of alternation, are quite similar in their broad outlines, there also exist numerous differences between the two languages. These differences have been the focus of the foregoing chapters.

Returning to the analysis in \cref{cha:pembrokeshire-welsh,cha:bothoa-breton}, we can identify at least three distinct sources of variation in sound patterns, all of which we could expect to find within the framework presented in \cref{part:theory}.

First, we find differences in the phonological grammars which drive patterns of distribution and alternation, as in the case of the behaviour of underlying \ipa{[ə]} in different Welsh dialects (\cref{sec:ot-analysis}), the difference between Welsh and Breton in the admissibility of complex onsets with glides (\cref{sec:analysis-9,sec:gliding}), or the lack of spreading from high vowels to preceding placeless stops (\cref{sec:palatalization}). This is the most traditional type of grammatical variation in OT, and one that is at least rhetorically claimed to be the only one necessary.

A second type of variation concerns the featural structure of \enquote{similar} sounds. For instance, while the stop systems of Breton and Welsh are extremely similar, there are significant differences in the representation of fricatives. In Breton voiceless fricatives differ minimally from voiced ones, which, in turn, are very simple segments, while in Welsh fricatives do not pattern as a phonological class at all: \enquote{voiced fricatives} are simple segments with no laryngeal counterparts, while \enquote{voiceless fricatives} demonstrate high segmental complexity, which is reflected in their phonological behaviour. This sort of variation is derived from the system of contrasts in the language, established on the basis of phonological behaviour (\cref{sec:featural-structure}): an old insight, recently revived in the shape of Modified Contrastive Specification \citep{torontoschool,dresher09,currie07} and fitting in well with the substance\hyp free approach \citep{moren-serbian,moren-foa,blaho-diss}.

Finally, a third, probably most interesting type, concerns the mapping between phonological representation and phonetic substance. The most prominent example of this sort of variation in this thesis is found in the area of laryngeal phonology. I have argued that both Welsh and Breton are best analysed as treating the \enquote{voiceless} or \enquote{fortis} obstruents as more marked than \enquote{voiced} ones, despite the phonetic differences between the implementation of this contrast in the two languages. Thus, I have rejected both the simplistic approach based on a single $[\pm\mbox{voice}]$ feature à~la \citet{wetzels01:_typol_voicin_devoic} and the phonetic essentialism of the \enquote{laryngeal realism} school \citep{iverson95,iverson99:_laryn_german,iverson03:_laryn_german,jessen02:_laryn_german,helgason08:_voicin_swedis,honeybone05}. I have argued that the similarity of the phonological patterns trumps the phonetic diversity, and thus that the source of cross\hyp linguistic variation in this particular instance is the essentially arbitrary mapping between phonological representation and phonetic substance.

This arbitrariness introduces a major source of uncertainty into attempts at comprehensive cross\hyp linguistic comparison (\cref{sec:emergent-features}). As discussed above, we can expect significant cross\hyp linguistic similarities in the types of phonetic cues used to implement phonological contrasts. This is especially so if \citet{kingston08} are correct in claiming that the integration of \enquote{raw} phonetic cues into what they call intermediate perceptual properties is not a specifically linguistic process, but rather one driven by the general properties of the human auditory system: in that case, the phonetic implementation of phonological contrasts is a kind of exaptation, taking a free ride on the properties of the auditory system that have no inherent linguistic relevance. However, the fact that a particular cue is used to implement a phonological contrast tells us precisely nothing about the specifically phonological characteristics of that particular contrast. Therefore, any major comparison of the sound patterns of different languages \emph{must} rely on an in\hyp depth analysis of the phonological systems of the languages, with particular attention to the system of contrasts and alternations and to the division of labour among the various components of grammar.

In this thesis I have proposed, by way of an extended example, a methodology for conducting this sort of analysis. I hope to have shown the importance of close attention to the affiliation of sound patterns as phonological or otherwise, the rôle of phonological contrast (and of its absence), and of the primacy of phonological patterning over phonetic substance. This has led to a reconsideration of the nature and importance of central notions such as \enquote{categoricity}, \enquote{contrastivity}, and \enquote{variation}. The substance\hyp free approach, grounded in the principle of modularity, has shown itself capable of providing explicit, independently founded criteria for resolving the numerous \enquote{boundary disputes} in current phonological thinking \citep{myers-phonoknowledge,cohn06:_is,cohn11:_featur,scobbie07:_inter}. Despite its commitment to a highly formal computation, substance\hyp free phonology with a rich phonetics\endash phonology interface promises the possibility of fruitful interaction with laboratory and variationist approaches, which rightly emphasize the subtly controlled nature of both \enquote{phonemic} and \enquote{subphonemic} sound patterns. At the same time, I have broached the possibility of integrating the substance\hyp free approach with Stratal OT, with its commitment to establishing independently grounded mechanisms for phonology\endash morphology interfaces and an explicit theory of the interaction of synchronic phonology and sound change. In the final reckoning, however, what I have emphasized most is the \emph{autonomy of phonology}, which still has a rôle to play in explaining the sound pattern of human language. The search for the boundaries continues, but I hope to have shown that phonology should not yet be annexed to other domains of the human knowledge of language.