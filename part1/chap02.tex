\chapter{Representational assumptions}
\label{cha:repr-assumpt}

In this chapter I present the representational system used in this thesis, which is a version of the Parallel Structures Model of feature geometry \citep{moren-psm,moren-serbian,moren-foa,kramer09:_italian,youssef09:_again_cairen_arabic,youssef10:_laryn_buchan_scots,iosad10:_motiv} that incorporates the insights of Modified Contrastive Specification \citep[\egm][]{torontoschool,dresher-hier,dresher09,ghini01:_asymm_miogl,ghini01:_place,dyck95:_const,currie07}. More specifically, I use the Parallel Structures Model (henceforth PSM), which is based on privative features, and adapt it to the Successive Division Algorithm (SDA), which \citet{dresher09} assumes to operate on binary features (see also \citealt{ghini01:_asymm_miogl,ghini01:_place,currie07} for versions of the SDA with privative features). I show that this version of the PSM allows us to combine the advantages of classic feature geometry (correct grouping of features that behave as a unit, explicit tier structure), language\hyp specific contrastive specification (adherence to the Contrastivist Hypothesis), privative features (economy, non\hyp stipulative expression of markedness relationships), and binary features (surface ternarity in phonology).

The chapter is organized in two sections. \Cref{sec:parall-struct-model} treats subsegmental structure, in particular the architecture of the Parallel Structures Model of feature geometry and its adaptation to Modified Contrastive Specification, while in \cref{sec:supr-struct} I briefly consider some issues related to suprasegmental phonology, arguing for a representational separation between the notion \enquote{head of a prosodic constituent} and \enquote{stress}.

\section{Segmental structure: the Parallel Structures Model}
\label{sec:parall-struct-model}

In this section I present the basic tenets of the Parallel Structures Model of feature geometry, proposed originally by \citet{moren-psm}. The PSM is a model based on unary features and an elaborate geometric structure that builds on the achievements of several previous theories. In its consistently privative approach to featural structure, the PSM is related to Particle Phonology \citep{schane84}, Dependency Phonology \citep{anderson87:_princ_depen_phonol,ewencoll}, and Element Theory \citep[\egm][]{harris94:_englis,harris95,cyran10:_compl,backley11:_elemen_theor}. The recursion of organizing nodes and the overall outlines of the treatment of place are inherited from Unified Feature Theory \citep{clements91:_place,clements91:_vowel_bantu,clements-hume1995}, while the treatment of manner has important points of contact with work such as that by \citet{lombardi90,steriade93:_oralit}.

The organizing principle of the PSM is \emph{economy}. It is a minimalist theory, in that it relies on a very small number of \emph{a priori} assumptions to derive universals of subsegmental organization (\ie restrictions on \textsc{Gen}; \citealt{moren-foa,uffmann07:_restr}), such as tier organization, node recursion, and a small number of (privative) features.  The number of such universals is consequently not very large, in particular since the phonetic realization of the structures created by the PSM mechanism is not job of the phonology. Nevertheless, there are also non\hyp trivial classes of potentially possible interactions between phonological objects that the PSM disallows: these are the impossible grammars that the theory bans (\cref{sec:enqu-gramm-relev}).

An important feature of the Parallel Structures Model is that features are never dependent on the root node itself: all features must be dominated by a class node, unlike some other proposals which treat at least major class features as dependents of root nodes \citep{sagey86,mccarthy88:_featur,halle95:_featur_geomet_featur_spread}. However, in the version of PSM I use in this thesis, the reverse does not hold: class nodes can be terminal, \ie a class node does not necessarily dominate a feature. Nevertheless, there is still a distinction between class nodes and features: the algorithm implementing contrast (described below in \cref{sec:bare-nodes-as}) is set up in such a way as to prevent class nodes from implementing lexical contrast in the absence of features. This makes the present version of PSM different from frameworks such as Element Theory and the proposal of \citet{blaho-diss}, which dispense with class nodes altogether and assume that features may simply depend on other features.

An example PSM representation in shown in \cref{fig:ex-repr}, and explained in more detail in the following sections. \begin{figure}[htp]\centering
\begin{tikzpicture}[parent anchor=south,child anchor=north,sibling distance=6em,xscale=1.2,yscale=1.4,on grid,text height=1.5ex,text depth=.25ex,level 1/.style={sibling distance=10em}]
\node {Root}
 child {node  {C-class node \ipa{Γ}} [clockwise from=-45,sibling angle=45]
child {node {[feature Y]}}
child {node {V-class node \ipa{Γ}}
 child{ node {[feature Z]}} child {node {[feature Y]}} child {node {[feature X]}} }
 child {node {[feature X]}}
}
 child {node {C-class node \ipa{Δ}} [clockwise from=-45,sibling angle=45]
child {node {[feature L]}}
child {node {V-class node \ipa{Δ}}
 child{ node {[feature M]}} child {node {[feature L]}} child {node {[feature K]}} }
 child {node {[feature K]}}
} ;
\end{tikzpicture}
\caption{An example PSM representation}\label{fig:ex-repr}
\end{figure}

\subsection{Tier organization}
\label{sec:tier-organization}

An important feature of the Parallel Structure Model of feature geometry is its commitment to tier structure. While SPE-style feature theories usually view segments as unordered bundles of feature values, in the PSM tiers are no less important than features, especially in the present version embracing the contrastive hierarchy. Tiers play two important rôles in this model, both of which have to do with restricting possible types of feature interaction: they sort features and they establish autosegmental domains. Before discussing these issues, I present a brief overview of the tier model.

\subsubsection{Tier structure in the PSM}
\label{sec:tier-structure-psm}

All features in PSM representations must be dominated by a class node, \ie no feature depends directly on the root node. A language can have several different types of nodes. Although in principle the PSM assumes that the labels associated with class nodes and features are arbitrary, for convenience I will use familiar labels such as Place, Manner, and Laryngeal, rather than, say \ipa{Α}, \ipa{Β}, or \ipa{Γ}. This is because (at least in the languages I concentrate on here) interactions between features can be described along these dimensions, although a single feature can have phonetic correlates along more than one of them. For instance, \citet{youssef10:_laryn_buchan_scots} argues that vowel height and consonant voicing in Buchan Scots are both expressions of a V-laryngeal feature, while in \cref{sec:spec-voic-fric} I suggest that a C-manner feature in Welsh corresponds to what would be traditionally seen as a bundle of features belonging to different types (non\hyp strident voiced fricatives).

A given feature can only depend on a class node belonging to one type: it is not possible for some feature to become reassociated from Manner to Place in the course of the derivation.\footnote{\citet{yip05:_variab} proposes that such reassociation should be possible, but her argument relies on cross\hyp linguistic comparison of the behaviour of the feature [lateral]. In the present framework, features in different languages are not comparable even if they have similar phonetic expression.} This is a restriction on \textsc{Gen}: since \textsc{Gen} is assumed to output only licit PSM representations \citep{moren-serbian,moren-foa,uffmann07:_restr}, the computation cannot enforce such a reassociation.

Another important aspect of tier structure in PSM is recursion. Class nodes, but not features, may dominate nodes of the same type: \ie a Manner node can dominate a Manner node (although not a Place node), while feature nodes are always terminal. Following standard PSM practice, I will refer to nodes dominated by the root node as C-nodes and to those dominated by a C-node as V-nodes. It must be emphasized, however, that this is purely a matter of convenience: there is nothing preventing consonants from having V-nodes, or vowels from having C-nodes: the affiliation of features depends on their phonological behaviour. It must also be noted that even though I only make reference to one level of recursion in this thesis (\ie there are no class nodes dominated by V-nodes), there is nothing in the representational system that prohibits their existence. They may well be required for some languages.

\subsubsection{Feature typing}
\label{sec:feature-typing}

When features are unordered bundles, there is in principle no restriction on how they may interact in phonological processes. In a geometrical theory, there exist representational restrictions on which features go under which tiers: for instance, there is widespread agreement that features such as [coronal] do not depend on class nodes such as Aperture or Manner. As discussed above, such restrictions are very strong in the PSM, since tier affiliation is essentially part of a feature's definition. I will call this aspect of the PSM \emph{strong feature typing}.

Assigning a type to every feature has a number of important consequences in terms of restrictions on possible feature interactions. Consider a situation where a class node spreads from one segment to another, but some aspect of the grammar of the language disallows the recipient segment to be or to become associated with the feature that this node dominates. In the PSM, there are only two solutions to this conundrum: either the ban is ignored or the spreading fails. In a representational theory without strong feature typing, a third solution is to spread the node but reassociate the offending feature to a different class node. The PSM makes the prediction that such processes should be impossible.

Note that it is the type of the class node that matters here, not its status as a C- or a V-node. This means that it \emph{is} allowed for a feature to reassociate from a C-node to a V-node or vice versa, as long as its type remains the same; for specific proposals to this effect, \cf \citet{clements91:_place,youssef11:_labial_baghd_arabic} and \cref{sec:front-rounded-vowel} below.

Strong typing also does not mean that features of different types do not interact at all. They may of course interact with each other, but this interaction is always mediated by their common mother node\dash most frequently the root node (\cf the discussion of *\{|A|, |B|\} constraints by \citealt[\S2.5]{blaho-diss}). It is possible for feature co\hyp occurrence constraints, for instance, to mention both Place and Manner features: however, they must formally refer to a node that dominates both of the relevant features. On the other hand, it is not possible for the presence or absence of, say, a Manner node to \enquote{count} when determining whether any potential targets have been skipped in a process involving a Place feature, which is a type of interaction that does not involve the root node. Again, this puts some non\hyp trivial (and thus falsifiable) restrictions on \textsc{Gen}.

\subsubsection{Locality}
\label{sec:locality}

Another task assigned to class nodes in classic autosegmental phonology is determining locality domains \citep[\egm][]{avery-rice,odden94:_adjac}: for instance, a segment lacking a Place node cannot be involved in a process spreading some Place feature, because it is invisible on the Place tier. Similarly, it is commonly assumed that autosegmental spreading cannot skip eligible targets, with tier structure used to determine whether a segment should be treated as such a target. The PSM inherits all these assumptions.

\subsection{Featural structure}
\label{sec:featural-structure}

Just like tier labels, featural labels are in principle arbitrary, although in practice more or less \enquote{phonetic} labels are used, for instance [coronal], [labial], and [dorsal] in the case of Place. Note that there is nothing in the theory to prevent us from incorporating other approaches to Place, such as \posscite{rice02:_vowel} [peripheral], should that be needed for some language, or even phonetically arbitrary \enquote{emergent} features.

The main function of features is implementing lexical contrast, in line with the Contrastivist Hypothesis. In this section I discuss the issue of featural economy and the unification between the PSM and the contrastive hierarchy that I use heavily in this thesis.

\subsubsection{Feature geometry and the contrastive hierarchy}
\label{sec:feat-geom-contr}

It is commonly acknowledged \citep[\cfm][]{christian-mfm16talk} that combining privative features with feature geometry appears to weaken the predictions of privative feature theory, in that it allows for surface ternary contrasts: where binary features allow $[\emptyset\mbox{F}]$, $[+\mbox{F}]$, and $[-\mbox{F}]$ and purely privative theories allow only $\emptyset$ and [F], a privative theory allowing bare nodes sides with the apparently less restrictive binary approach, allowing \featurestring{\rt}, \featurestring{\rt, Node}, and \featurestring{\rt, Node, [F]}. This would seem to be a major weakness of geometric approaches, and consequently many authors stipulate that bare nodes are not possible in the representational system \citep[\egm][]{lombardi95:_book}, ostensibly because representations such as \featurestring{\rt} and \featurestring{\rt, Node} never contrast with one another in a single language.

However, it has also been pointed out that ternarity is in fact empirically necessary. One type of argument to this effect was adduced in underspecification theory, where the lack of specification for a feature is a crucial factor in the analysis. However, a major drawback of these approaches is that they usually assume a fully specified surface representation, and thus the importance of ternarity is essentially analytic. A more convincing argument for the necessity of ternary representations is the existence of cases of \emph{surface ternarity} \citep{kim02:_phonol,strycharczuk12:_phonet}, which appear to falsify the strictly privative approach.

In this thesis I present a case of surface ternarity from a dialect of Breton\footnote{Breton facts have been used to motivate ternary contrasts by \citet{kramer-breton} in a binary\hyp feature framework and by \citet{hall09:_laryn_breton} in a geometric theory (with bare nodes), but both of these authors rely on ternarity in underlying representations rather than on the surface. Their work is discussed in detail below (\cref{sec:mark-relat-bret}).} and argue that ternary contrasts must indeed be expressed in geometrical terms. Following \citet{ghini01:_asymm_miogl,ghini01:_place}, I treat representations with bare nodes as the result of contrastive non\hyp specification for a privative feature. In this thesis I demonstrate that while this approach is less restrictive than one based on strictly unary features, it is \emph{more} restrictive than a binary\hyp feature framework, since it has a number of additional implications that are unavailable in other feature theories without additional stipulation.

\subsubsection{Bare nodes as contrastive non\hyp specification}
\label{sec:bare-nodes-as}

In this thesis I use a version of the Parallel Structures Model where representations with bare nodes are possible both in input and output representations. The difference between featureless representations with and without bare nodes is related to contrast.

I suggest that learners are biased to posit inventories that are consistent with a contrastive hierarchy built up by \posscite{dresher-hier} Successive Division Algorithm \citep[§7.8]{dresher09}. I use a version of the SDA that is similar to that proposed by \citet{ghini01:_asymm_miogl,ghini01:_place} but adapted to the representational system of the PSM. Specifically, I assume that at each cut of the inventory some subset of that inventory becomes associated with a feature, and therefore\dash  given the architecture of the PSM\dash with a class node. I propose that the complement of this marked set receives the bare class node.

\begin{figure}[htp]
  \centering\setlength\level{15mm}
  \begin{tikzpicture}[hierarchy,level 1/.style={sibling distance=10em}]
    \node {\ipa{[i u a]}}
      child {node {V-man[cl] \\ \ipa{[i u]}}
        child {node {V-pl[cor] \\ \ipa{[i]}}
          child {node {\ipa{/i/} \\ \{V-pl[cor], V-man[cl]\}}}}
        child {node {V-pl \\ \ipa{[u]}}
          child {node {\ipa{/u/}\\\{V-man[cl]\}}}}}
      child {node {V-man \\ \ipa{[a]}}
        child[level distance=2\level] {node {\ipa{/a/} \\ $\emptyset$}}};
  \end{tikzpicture}
  \caption{Feature geometry as contrastive non\hyp specification}
\label{fig:contrastive-hierarchy-psm}
\end{figure}

A toy example of this procedure is shown in \cref{fig:contrastive-hierarchy-psm}.\footnote{For extended examples, see \cref{fig:welsh-hierarchy,fig:bothoa-contrastive-hierarchy}.} The three\hyp vowel inventory \ipa{/i~u~a/} can be classified in a number of ways. For the sake of the argument, I use a contrastive hierarchy which puts \fea{C-manner}{closed} (or, more traditionally, $[(\pm)\mbox{high}]$) above \fea{V-place}{coronal} ($[\mbox{coronal}]$ or perhaps $[(\pm)\mbox{back}]$). Under this contrastive hierarchy, \ipa{/i/} is treated as \{\fea{V-pl}{cor}, \fea{C-man}{cl}\}, \ipa{/u/} is \fea{C-man}{cl}, and \ipa{/a/} remains featureless. I propose, however, that contrastive non\hyp specification is reflected by tier structure: thus, despite being featureless, \ipa{/a/} in this system does bear an empty V-manner node, and \ipa{/u/} bears an empty V-place node. Crucially, however, \ipa{/a/} does not bear a V-place node, because it does not contrast for V-place features.

Thus, tier structure essentially recapitulates the key insight of underspecification theory, in making a distinction between lack of featural specification that is due to a lack of contrast and lack of specification as the consequence of contrastive feature assignment. This distinction is very easy to express using binary features as one between $[-\mbox{F}]$ and $\emptyset$, but it is unavailable in theories using privative features. However, there are several additional implications of this approach, discussed in \cref{cha:categ-contr-mark}, that are not expressible in theories using multiply valued features.

The crucial point here is that the presence of a class node signifies the \emph{existence of contrast} along some dimension, while its absence signifies the lack of (phonological) contrast. Note, however, that in this system nodes themselves cannot be used to implement lexical contrasts, because, per the algorithm, a class node can only appear in the representation if the assignment of some feature requires it.

The presence or absence of structure can be due both to the properties of the lexicon (\ie the presence of lexical entries containing segments that necessitate the contrast) and to the computation. For instance, since class nodes and features are similar phonological objects, we can posit that markedness constraints of the *[F] family may target both of them. In this case, a constraint (say) *V-place can be used to neutralize all V-place contrasts by enforcing (given the correct ranking) the deletion of the V-place node and thus all its dependent features (\citealp[\cf][]{ghini01:_place} for a concrete implementation). Importantly, since the computation is free to manipulate phonological representations without reference to properties of the input, it can also create output structures that are not needed for lexical contrast or not consistent with a contrastive hierarchy. Thus, in \cref{cha:bothoa-breton} I argue that although only two classes of obstruents are required for lexical contrast in Bothoa Breton, the computation creates a third class of \enquote{delaryngealized} obstruents, which are identical to laryngeally specified obstruents but lack a C-laryngeal node.

This autonomy of computation, \ie its relative freedom to manipulate phonological representations, is a major source of cross\hyp linguistic variation. It also shows that the contrastive hierarchy, despite its important rôle, is not enough \emph{per se} to account for differences in phonological patterning across languages. In particular, the computation may introduce structures that are at odds with the contrastive hierarchy. What this means is that the hierarchy is essentially a bootstrapping device, which allows the learner to introduce order into the system of phonological contrasts by breaking the phonological space down into more manageable subinventories. In that sense, it serves purposes that are highly similar to those claimed for the concept of feature economy \citep[\egm][]{clements03:_featur}.

This view of the contrastive hierarchy allows us to reject \posscite{currie07} conclusion that it is incompatible with an OT approach, and specifically with Richness of the Base. Even if the learner converges on a lexicon where all entries are made up of segments that can be arranged into a contrastive hierarchy, it is still incumbent on the computation to map inputs for which fully faithful candidates are disharmonic to allowable outputs. Since the contrastive hierarchy is \emph{not} construed as a restriction on possible inputs (unlike the principles of the representational system, \ie the PSM), a restricted version of Richness of the Base is still upheld.

\subsubsection{The problem of empty segments and hierarchy subversion}
\label{sec:probl-empty-segm}

One apparently undesirable feature of marrying the contrastive hierarchy with a privative approach is that a privative version of the SDA will always designate one segment as being featureless \citep{currie07,blaho-diss}, as is the case with \ipa{/a/} in \cref{fig:contrastive-hierarchy-psm}. However, I would suggest this is not necessarily a fatal problem. The solution to this issue is partly representational and partly computational.

From the representational perspective, there is no logical requirement for empty root nodes to be impossible segments. Empty root nodes are possible in surface representations in a variety of theories, most prominently in versions of CVCV phonology \citep[\egm][]{scheer04}, although they are also found in other frameworks; to take a random example, empty (unpronounced) root nodes play a crucial rôle in \posscite{kohnlein11:_rule} analysis of the prosody of the Arzbach dialect. Crucially, we even find examples of featureless root nodes that are \emph{pronounced}, for instance as a schwa \citep[\egm][]{oostendorp00:_phonol,nesset-reduction}. In this sense, it is not entirely clear that the prediction of the existence of a featureless segment is necessarily incorrect for the language at stake: there may well be good evidence for such a representation.

From the computational perspective, it is important that the contrastive hierarchy serves as a device to construct plausible inputs, not to construct the full set of possible outputs. Thus, it is fully possible for the computation to map an input empty segment to something else, especially if there is some evidence for that in the patterns of alternation (we shall see some evidence for that in Breton in \cref{sec:analysis-4}). In this case, just as in the previous one, the existence of the featureless segment in the input to the phonology has no significant consequences for the surface inventory.

Finally, given that the contrastive hierarchy is not construed here as the only source of feature specifications, it is also logically possible that the place of the featureless segment on the hierarchy could be unoccupied, or exceptionally occupied by a segment that is not specified in line with the version of the SDA used here. This latter scenario is especially likely when phonological evidence forces the learner to posit some segment which cannot be accommodated by the hierarchy at hand. I propose that this is the case for Pembrokeshire Welsh \ipa{[ŋ]} (\cref{sec:representations}).

This latter problem forces us to confront the issue of whether phonological evidence can lead the learner to construct inventories that are not fully in line with the restrictions on inventories available in the input \citep[as argued by][]{blaho-diss,kramer09:_italian}. The answer would seem to be positive: the contrastive hierarchy as construed here is a bootstrapping device or a bias to organize the system of contrast, not an absolute restriction on inputs, and given the autonomy of computation it should not be problematic for some features or feature configurations to be preserved on the surface despite not fitting in with the hierarchy. This allows for both minor deviations (as with Pembrokeshire Welsh \ipa{[ŋ]}) and, in principle, systems built without much regard for the SDA, as in \citet{blaho-diss} and \citet{kramer09:_italian}. Nevertheless, as a heuristic, I suggest that a solution that cleaves more closely to the hierarchy is, in general, to be preferred to a less structured one, at least if \citet{dresher09} is right in his approach to SDA\hyp driven phonological acquisition.\footnote{At this point this must remain an æsthetic judgement, although in principle it should be possible to test two competing substance\hyp free hypotheses with and without the SDA (\eg using psycholinguistic techniques). This would require constructing the two analyses first, which  is far outside the focus of the present thesis.}

\subsubsection{Further consequences of gradualness}
\label{sec:furth-cons-grad}

The contrastive hierarchy as a way of organizing the system of contrast is an alternative to \citeauthor{moren-psm}'s \citeyearpar{moren-psm,moren-serbian,moren-foa} proposals regarding the gradual structure of representations. He suggests that learnability requires all complex featural structures possible in a language to have the property of being divisible into two possible simple structures. A corollary of this principle is that all features possible in simpler structures should have a \enquote{unit segment}, \ie a segment consisting just of that feature, because a complex structure \{|A|, |B|\} must by this hypothesis be divisible into \{|A|\} and \{|B|\}, \cf \citet{blaho-diss} for extensive discussion. An inventory such as that shown in \cref{fig:contrastive-hierarchy-psm} should be impossible in this version of the PSM, since \fea{V-place}{coronal} does not have a unit segment in this toy example.

I would suggest that the contrastive hierarchy offers an alternative to \citeauthor{moren-psm}'s conception of gradualness, since it also allows to build up bigger structures from smaller ones, without necessarily requiring that there should exist a unit segment for each feature. In practice most features will still have unit segments, because, as noted above, the privative version of the SDA will always specify one segment in each (sub)inventory as featureless. Thus, there will always be one segment that does not receive a feature. Once a feature is used to make a cut in the inventory, there will exist a segment which possesses that feature but no others.

A further restriction on the shape of inventories, noted by \citet{blaho-diss} and holding of the version of PSM proposed here, is that the gradual build-up of contrastive structure is reflected in dependency relations between phonological elements. In terms of PSM this means that when, say, a Manner feature is used to divide a set of segments that have not yet been specified for Manner, the resulting node will always be a C-manner. On the other hand, if the relevant segments have already received a Manner node at a previous iteration of the SDA, the relevant feature can either be added to the existing C-manner node or to a recursive, \ie V-manner node. In other words, if the language makes use of a C- and a V- tier for some dimension, at least one C-feature must be higher than all V-features on the hierarchy.

This concludes the discussion of the representational assumptions used in the present thesis to account for segmental patterns. In the next section provide a brief account of some representational aspects of suprasegmental phonology in a substance\hyp free approach that are relevant to the analysis that follows, with particular reference to the representation of stress.

\section{Suprasegmental structure}
\label{sec:supr-struct}

In this thesis I do not focus on issues in suprasegmental phonology to a very large extent, with the exception of \cref{sec:vowel-mutation} below. In this section I discuss a particular proposal for the representation of \enquote{stress} in (some) languages that is in line with the tenets of substance\hyp free phonology. I suggest that emergent features with no firm phonetic grounding are found not only in segmental phonology but also within the prosodic domain.

\subsection{Suprasegmental features}
\label{sec:supr-feat}

When discussing the issue of features in prosody, we are faced with two questions. First, are there suprasegmental features different from those found in subsegmental phonology? Second, can features (of either type) attach to prosodic nodes or are they confined to the domain below the root node?

I suggest that the answer to the first question is a qualified \enquote{yes}: there is no significant difference in kind between suprasegmental features such as tone and subsegmental features such as manner, although the former, unlike the latter, are not always used to implement lexical contrast. (I will return to this issue in \cref{sec:contrast-stratal-ot}.) As for the second question, I suggest that features may indeed attach to higher\hyp order prosodic constituents, with important segmental consequences.

The suprasegmental feature \emph{par excellence} is of course tone, and, if suprasegmental features are not any different from subsegmental ones, we could assume that they have much the same geometrical structure as that sketched in \cref{sec:parall-struct-model} (see \citealt[§3.4]{yip02:_tone} for an overview of some geometrical approaches to tone). There is also considerable evidence for the interaction between tones and segmental features \citep{hyman74:_univer_tone_rules,hombert78:_conson,jiang-kang99:_tone_optim_theor,bradshaw99,tang08,beckerng:_inter_atr_sloven}, which has been formalized \citep[\eg by][]{bradshaw99} in terms of a tone feature attaching to different geometrical nodes (\eg Laryngeal for consonants and Tone for tones). Since this is impossible in the PSM (\cref{sec:feature-typing}), the distinction between supra- and subsegmental varieties of a single feature must be treated in terms of attachment to different nodes in the hierarchical structure.

The attachment of normally subsegmental features to higher\hyp order hierarchical nodes is not a novel proposal either, see \eg \citet{lodge93:_under,Lodge2003931,lodge07:_timin_icelan,kehrein,kehrein04}. In both of these cases we have to assume that the features attached to some node above the segment can percolate down to the root node level and interact with subsegmental features just as any others. In this situation, there is nothing in the substance\hyp free approach that precludes positing emergent features in the suprasegmental domain similar to those argued for in subsegmental representations. If the existence of such features is accepted, we are in a position to treat a range of phenomena in terms of such emergent features. In the next section I briefly consider the issue of the nature of \enquote{stress}.

\subsection{Stress and headedness}
\label{sec:stress-headedness}

I suggest that in some languages \enquote{stress} is precisely one such suprasegmental feature, which is ontologically distinct from prosodic headship. The tight fit between heads of prosodic constituents and bearers of the stress feature is a typologically frequent, but not exceptionless effect. Although the idea is not new by any means \citep[\egm][]{crowhurst-hewitt1995,hyde,hyde06:_towar,vaysman08,buckley09:_local}, I suggest that conceptualizing stress as a feature has a number of important implications.

Following work such as that by \citet{dresher-vdhulst}, I view \emph{grammaticalized asymmetry} as a defining characteristic of prosodic organization \citep[\cf also][]{curt-diss,rice07,weijer96:_segmen,mellander04:_hl}. In other words, the primary property of the head of a prosodic constituent is that it \emph{may} have some properties that a non\hyp head does not have; it is a \enquote{strong} position \citep[\egm][]{smith-diss,smith04:_makin,teeple09:_bicon}. For instance, head constituents may be required to have more branches than their sister non\hyp heads \citep{dresher-vdhulst,mellander04:_hl}, or they may license features that are disallowed in non\hyp head position (\citealp{harris97:_licen_inher,harris05:_vowel_reduction,harris-urua}, \cf also \citealt{iosad10:_motiv}).

It is very common cross\hyp linguistically for elements (\eg syllables) demonstrating these asymmetric properties to also bear some sort of phonetic prominence, \ie \enquote{stress}, which is often formalized in terms of the metrical grid \citep[\egm][]{prince1983,halle-vergnaud,hayes1995,hyde}. An often\hyp repeated claim in the literature is that headship and stress are in fact the same thing, \eg by \citet{halle-vergnaud}. Nevertheless, there is some evidence that this is not necessarily so.

If headship and stress are logically independent, we might expect there to exist mismatches, \ie situations where some heads surface without stress and some instances of stress may surface on non-heads. Both situations appear to be attested.

\subsubsection{Unstressed heads}
\label{sec:unstressed-heads}

One type of mismatch involves situations where asymmetry considerations lead us to expect the presence of a head, but where these heads do not show the phonetic characteristics of \enquote{(secondary) stress}. A celebrated case is Cairene Arabic \citep[\egm][]{hayes1995}, where iterative footing is necessary to achieve correct placement of main stress but where secondary stress is claimed to be absent. If all feet are headed (a common if not universally accepted assumption), this means that the heads of all feet except the head of the word are not stressed. Similar examples are found in Kera \citep{pearce06:_kera}, where foot heads demonstrate special behaviour in terms of phenomena such as tone placement and vowel harmony, but lack \enquote{secondary stress}, and in Latvian (\citealt{buckley09:_local} citing \citealt{karins96:_latvian}, although see below \cref{sec:latv-evid-bimor} and \citealt{daugavet05:_palig} for more discussion).\footnote{Sam Hellmuth (p.~c.) points out that Cairene Arabic may well be similar to these languages, in that headship in subsidiary feet is probably cued by other means than \enquote{stress} in that language as well.}

In standard derivational theory, this situation can relatively easily be accounted for using the device of tier conflation, if it follows all processes crucially depending on the head status in subsidiary feet. In this case, the lack of unstressed heads of this type is not a significant problem. However, in parallel constraint\hyp based approaches this type of opacity requires special explanation. Some of these cases could perhaps be assimilated to a stratal solution, but I would suggest that it is also possible to account for head effects in the absence of stress representationally \parencite[\cf][]{crowhurst96}. Specifically, in these cases head status is marked in the output of the phonology, and is responsible for relevant effects in a transparent manner, but the computation fails to associate a prominence feature, or \enquote{stress}, to heads of (some or all) metrical feet.

More compelling evidence, however, comes from the existence of stress on non\hyp head constituents.

\subsubsection{Stress on non\hyp heads}
\label{sec:stress-nonhyp-heads}

Most of the cases in this rubric come in the form of mismatches between the foot structure hypothesized on the basis of processes other than stress assignment and that required for metrical processes, as in \citet{downing06,vaysman08}, although similar proposals have also been made purely on the basis of metrical phenomena \citep{hyde,hyde06:_towar,iosad08:_compl}. The importance of these cases lies in the fact that they show the presence of \emph{both} stress and headship on different elements in the same language, confirming the independence of these two phenomena.

To illustrate this, consider the interaction of \textitalian{\emph{raddoppiamento fonosintattico}} and stress retraction in Roman Italian \citep{garvin,kramer09:_italian}. In this dialect \textitalian{\emph{raddoppiamento}}, a process that adds a mora to a word\hyp final stressed syllable by geminating the first consonant of the following word, is counterfed by clash retraction. Thus, while many varieties of Italian allow clash in cases such as \ipa{[kaf(ˈfɛl) (ˈluŋ)ɡo]} `diluted coffee' (\ipa{[kafˈfɛ]} `coffee', \ipa{[ˈluŋɡo]} `long'), in Roman Italian we find \ipa{[(ˈkaf)(fɛ\emph{l}) (ˈluŋ)ɡo]}, with clash-avoiding retraction accompanied by overapplication of the gemination rule. \citet{kramer09:_italian} analyses this case of opacity using \enquote{headless feet} and output\endash output correspondence constraints. However, the second foot in \ipa{[(ˈkaf)(fɛl) (ˈluŋ)ɡo]} is \enquote{headless} only because there is no stress. In fact, it does demonstrate behaviour characteristic of heads, since the doubling is presumably due to a requirement for the head foot of the word to be branching \citep{dresher-vdhulst}. We can thus analyse this case as a transparent interaction of prosodic structure (in this case, lexically stored footing\footnote{Word\hyp final stress in Italian is unpredictable and must be stored somehow in any case \citep{italianstress,kramer09:_italian}.}) and the assignment of the stress feature. As an aside, if stress is a feature\hyp like entity, the constraint \textsc{*Clash} motivating retraction in Roman Italian is just another guise of the obligatory contour principle.

Thus, cases such as Roman Italian show that divorcing the prosodic status of certain constituents as heads from the notion of stress is not just a conceptual possibility, but in fact a useful feature in the analysis of attested phonological phenomena.

\subsection{Emergent suprasegmental features?}
\label{sec:emerg-supr-feat}

In the previous section I argued that the substance\hyp free approach permits us to view \enquote{stress} as an (emergent) feature, which can be manipulated by the phonological computation independently of prosodic headship. In this sense, then, \enquote{stress} is just another substance\hyp free feature, without significant ontological differences from other features. The fact that is often appears only in head position is, from a purely computational perspective, simply parallel to the fact that certain unfaithful mappings are blocked in head positions, as with vowel reduction or \enquote{foot\hyp internal} lenition à~la \citet{harris97:_licen_inher}. The substance\hyp free nature of \enquote{stress} is further buttressed by the fact that different languages in fact choose different strategies to express it phonetically \citep[\egm][]{hulst99:_word}.

It must be emphasized that I view the \enquote{stress} feature in these cases as a completely abstract entity. Most importantly, it does not equal the tone features which may be associated with certain designated \enquote{pivots} by the postlexical phonology as part of the system of intonation (the starred tones in standard notation). I hypothesize that the relationship between these pivots and the abstract \enquote{stress} features is also regulated by the phonological computation. For more discussion of this issue in the context of a more concrete example, see below \cref{sec:pros-prom-as}.

If such emergent, substance\hyp free features are allowed to coexist with more phonetically grounded ones such as tone, can we expect \enquote{monster features} with no consistent phonetic expression? In principle, there is nothing in the theory to prohibit this. However, this is not necessarily an undesirable prediction: in \cref{sec:pros-prom-as} I propose just such an abstract feature to account for a number of phonological patterns in Pembrokeshire Welsh, following work by \citet{bosch96:_promin}. Finding further examples of such features with good phonological motivation but a rather unclear phonetic rationale will remain a task for the future.

Finally, to avoid further confusion, I will only use \enquote{stress} in the remainder of this thesis to refer to \emph{prosodic heads}, \ie the loci of prosodic asymmetries, and never to the abstract feature that may be associated with (some of) these heads. Again, see \cref{sec:pros-prom-as} for more discussion of this issue.

This concludes the discussion of the most important representational assumptions made in this thesis. I have not raised issues that will be relevant to the analysis but where my position does not differ significantly from that taken in previous literature; such questions are discussed below as necessary. In the next chapter I offer an account of some of the most significant computational proposals that this thesis rests on, in particular the rôle of licensing, or enhancement, constraints, and stratal phonological computation.