\chapter{Categoricity, contrast, and markedness}
\label{cha:categ-contr-mark}

In this final introductory chapter I take up some threads of previous discussions, with specific attention to three topics. First, I argue that the existence of categorical distributions cannot be taken to define the phonological status of a given phenomenon. Second, I show that, given a relatively powerful computation, paradigmatic contrast (\ie the existence of \enquote{minimal pairs}) is a sufficient, but not necessary criterion for establishing the existence of a phonological distinction (\ie a difference between phonological symbols). Third, I discuss the relationship between markedness and substance, and argue that the structural markedness approach coupled with a substance\hyp free framework provides the correct compromise between inherentist and emergentist theories of markedness in phonology.

\section{The relevance of categorical distributions}
\label{sec:relev-categ-distr}

A frequent criterion used to distinguish between \enquote{phonetics} and \enquote{phonology} is the difference between \enquote{gradient} and \enquote{categorical} patterns \citep[\egm][]{myers-phonoknowledge}, although these terms themselves are not entirely unambiguous \citep{cohn06:_is,scobbie07:_inter}. In particular, the term \enquote{categorical} appears to be used whenever the distributions demonstrate any sort of bi- or multimodality; thus, any distinctions that cannot be described with a single continuous distribution can be proclaimed \enquote{categorical} and therefore \enquote{phonological}, see \eg the discussion in \citet{tucker10:_what}.

However, using the existence of \enquote{categorical} distributions in the data to derive conclusions about the nature of that data is beset with difficulties. For instance, identifying what appear to be two modes does not necessarily indicate that the underlying distribution is bimodal, especially if the two modes are close to each other \citep{schilling02:_is_human_heigh_bimod}, and this is often the case with finely grained phonetic data. Moreover, as pointed out by \citet[§1.5]{scobbie07:_inter}, distributional discontinuities at a finer level of detail than that needed to describe lexical contrast are all but inevitable, given that some of conditioning factors for phonetic variation are inherently discrete.

The crux of the matter is that, in the final reckoning, in the absence of very tight controls, data including categorical or gradient distributions are \emph{behavioural}, \ie they describe E-language rather than I-language \citep[\cfm][]{lacy09:_phonol}. This point is important not just for methodological reasons, but because the literature does in fact contain examples where pooled data show more clearly categorical distributions than data for single speakers \citep{padgett-tabain,scobbie06:_flexib_englis_vot,scobbie07:_inter}; conversely, \enquote{gradient} phenomena may show interesting categorical differences among individual speakers \citep{ellis02:_categ}. If theoretical phonology aspires to describe the individual's knowledge of language, we should not take categorical \emph{behaviour} as primary evidence for categorical \emph{knowledge}.

Another problematic issue is the existence of a pattern involving apparently random choice between what appear to be categorical variants \citep{baayen11:_mixed}, which defies neat classification as \enquote{categorical} or \enquote{gradient} \citep[for examples, see][]{scobbie09:_dutch,mielke10:_variab_americ_englis,strycharczuk11:_explain,bermudez-oterong:_cycles}. This pattern is discussed in more detail below in  \cref{sec:recons-surf-undersp}.

The key point of this brief discussion is that categoricity cannot be used as a \emph{defining} criterion of phonological status. In \posscite{scobbie07:_inter} classification, the approach used in this thesis pushes the division between phonology and phonetics very high towards the phonological end, defining phonology as categorical operations on phonological symbols. This leaves much of the discontinuous, language\hyp specific variation in sound patterns outside the phonology, in what I call the interface. In \cref{sec:interf-as-interpr} I posited some restrictions on the operation of the interface, in particular its inability to collapse or introduce categorical distinctions. Note, however, that this does not in any way preclude the appearance of \enquote{categorical distributions} (\ie multiple modes) in the data. Within the range of variation in the realization of a phonological category, some of the options can be favoured for functional, mechanical, or social reasons, creating statistical significance. This does not mean that the underlying interpretational mechanism \emph{per se} produces categorical distinctions. For instance, instrumental studies of Russian vowel reduction \citep[\egm][]{padgett-tabain} have demonstrated a clear, statistically significant difference between the allophones of unstressed vowels depending on their position relative to stress, and the theoretical literature \citep[\egm][]{crosswhite} has also treated this allophony as involving at least two phonological categories (\eg\ \ipa{[a]} and \ipa{[ə]}): however, both phonetic and phonological analyses \citep{barnesbook,barnes2007,iosad10:_motiv} have shown that there is no \emph{phonological} difference behind the statistical significance.

Thus, if phonology is defined as computation over phonological symbols, categoricity is not \emph{sufficient} to define the phonological status of certain phenomena. Conclusive evidence for phonological status should comprise evidence of categorical symbolic manipulation, thus categoricity\dash assuming it can be extracted from the data\dash is a necessary, and often suggestive, piece of such evidence. Nevertheless, since it does not \emph{define} what is phonological and what is not, it is not sufficient to identify phonological patterns. The question of how we can identify symbols as phonological is the subject of the next section.

\section{The status of contrastivity}
\label{sec:status-contrastivity}

As discussed in \cref{cha:subst-free-phon}, the present thesis uses a version of the Contrastivist Hypothesis, which \citet[p.~20]{currie07} states as follows: \blockquote{The phonological component of a language L operates only on those features which are necessary to distinguish the phonemes of L from one another.} Crucially, as stated here the Contrastivist Hypothesis does \emph{not} require that the phonological component operate with features that are \emph{sufficient} to distinguish the phonemes of a language from one another.

The crux of the matter is the status of predictable distributions. In the presence of lexical contrast, as established by \enquote{minimal pairs}, the distinctive status of the relevant features (assuming for the moment some algorithm for extracting them) is not in doubt. However, the phonological status of distinct units with a predictable distribution can be difficult to establish. As \citet[p.~29]{scobbie07:_inter} emphasizes, \enquote{equating [a] phoneme $\Phi_{1}$ and a [\ldots] phoneme $\Phi_{2}$ via an allophonic relationship does not in any way define the allophonic relationship itself as either phonological or phonetic}. This is especially true in a theory such as the present one, where the phonological component is given free rein to establish any computable distribution of the available representations. In this section I will argue that nothing in the theory forces us to excise predictable information from the phonological component, and in particular that elements which appear superfluous for the purposes of lexical contrast may still be treated as potentially possible in lexical representations, and therefore available to phonology.

\subsection{Establishing predictable distributions in the phonology}
\label{sec:establ-pred-distr}

In a sense, the argument in this section recapitulates the argument of \citet{spr}: if some segment is an \enquote{allophone} of some other phoneme, in the sense of having a predictable distribution, it does not automatically mean that this distribution is not due to the same phonological factors as a similarly predictable distribution of elements that are otherwise contrastive. On the other hand, \posscite{spr} way out of the conundrum\dash treating all language\hyp specific sound patterns as phonological\dash cannot be accepted either: while he argued against treating similar alternations as belonging to different components of grammar, the existence of rule scattering (\cref{sec:rule-scattering}) shows that this argument is not necessarily correct in all cases.

An even more vexatious issue is the treatment of distributions that are predictable relative to non\hyp phonological factors, as in the case of Northern Irish English contrast between \ipa{[ɪə]} and \ipa{[ɛː]} \citep{harris94:_englis}: normally, the former appears in closed syllables and the latter in open ones (hence they are \enquote{the same phoneme}), but in certain morphological contexts the complementary distribution breaks down, giving rise to what would appear to be minimal pairs, as in \ipa{[ˈdɪəz]} `daze' but \ipa{[ˈdɛːz]} `days' (\cf\ \ipa{[ˈdɛː]} `day', *\ipa{[ˈdɪə]}). The distinction between \ipa{[ɪə]} and \ipa{[ɛː]} could be treated as non\hyp phonological (\ie \enquote{phonetic}), but stating the distribution of the allophones would require recourse to morphological constituency, in violation of modularity. The alternative account, much closer in spirit to the present approach, of course involves a combination of a ranking that establishes the complementary distribution between phonological \ipa{[ɪə]} and \ipa{[ɛː]} at the stem level and a second cycle of (word\hyp level) phonology that preserves the distribution established earlier.

\begin{figure}[htp]
  \centering
  \begin{tikzpicture}[>=latex,line join=bevel,]
%%
\node (Faith-e) at (269bp,18bp) [draw,rectangle] {\textsc{Faith}(\ipa{ɛː})};
  \node (*ie) at (190bp,18bp) [draw,rectangle] {\ipa{*ɪə}};
  \node (*ec) at (229bp,90bp) [draw,rectangle] {*\ipa{ɛːC\textrbracket\ssy}};
  \node (Faith-ie) at (36bp,18bp) [draw,rectangle] {\textsc{Faith}(\ipa{ɪə})};
  \node (*e) at (118bp,18bp) [draw,rectangle] {*\ipa{ɛː}};
  \node (*iesf) at (77bp,90bp) [draw,rectangle] {*\ipa{ɪə}\textrbracket\ssy};
  \draw [] (*ec) -- (*ie);
  \draw [] (*iesf) -- (Faith-ie);
  \draw [] (*ec) -- (Faith-e);
  \draw [] (*iesf) -- (*e);
%
\end{tikzpicture}
  \caption{Stem-level schematic ranking for Northern Irish English}
  \label{fig:days-daze-ranking}
\end{figure}

Establishing complementary distribution requires a relatively complex ranking that is able to cope with counterfactual inputs supplied by the rich base: for instance, in the case of Northern Irish English it would have to map both \ipa{/dɪə/} to \ipa{[ˈdɛː]} (or to some other phonotactically correct candidate) and input \ipa{/dɛːz/} to \ipa{[ˈdɪəz]}. Crucially, deriving a case such as this does not require us to make any commitments with respect to the ranking of the markedness and faithfulness constraints for the elements which stand in complementary distribution (\ipa{[ɛː]} and \ipa{[ɪə]}) in this case: all that is required is that the contextual markedness constraints are undominated. The ranking shown in \cref{fig:days-daze-ranking} (with grossly simplified constraints) is sufficient to derive the complementary distribution, and it does not \emph{require} any ranking between context\hyp free markedness (\ipa{*ɪə}) and faithfulness (\textsc{Faith}(\ipa{ɪə})).

It is, however, clear that the phonology must be able to refer (at least) to the representation \ipa{[ɛː]}, because, at later levels, it is available to faithfulness constraints preserving it in \ipa{[ˈdɛːz]} `days', presumably by reranking of \textsc{Faith}(\ipa{ɛː}) over \ipa{*ɛːC\textrbracket\ssy} at the word level. The upshot of this discussion is that the phonological computation is perfectly capable of enforcing a predictable distribution (otherwise known as \enquote{lack of contrast}) without the elements involved losing their phonological status.

This situation does not in any way undermine the Contrastivist Hypothesis. We may well still assume that the lexicon in Northern Irish English contains the entries \ipa{/dɛː/} for `day' and \ipa{/dɪəz/} for `daze' (and others like them), despite the fact that there is a complementary distribution to be extracted (after all, the lexicon is most often treated in generative models as a graveyard for lost generalizations). Indeed this is the expected outcome of the life cycle of phonological processes \citep{bermudez-diachr,kaisse11:_lexic_phonol,bermudez-oterong:_cycles}: since the complementary distribution clearly holds at the stem level, the lexicon is obviously the next step on the way.

These cases demonstrate that evidence for the phonological status of a phenomenon cannot come simply from the distribution. The missing factor here is computation, which can enforce both predictable and unpredictable distributions of the elements. If phonology is defined as categorical computation over phonological symbols, and the computation in a particular case is clearly phonological\dash as here, where it involves the stratal architecture that is the privilege of phonology\dash the symbols it involves must by definition be phonological. Therefore, the best evidence for some distinction having phonological status is its participation in what can be shown to be phonological processes, and in particular alternations.

\subsection{Further examples of predictable phonology}
\label{sec:furth-exampl-pred}

The prime example of phonological phenomena involving predictable distributions is found in prosodic phenomena such as syllabification. It is generally acknowledged that at least syllabic structure is usually predictable, \ie not used for lexical contrast (although \cf~\citealt{vaux-syll}), but clearly visible to the phonological computation in a variety of ways. Similarly, it is commonly acknowledged that in many languages moraic and foot structure are not necessary in underlying representations, and built by the computation (\cf \citeauthor{moren01:_distin}'s \citeyear{moren01:_distin} \enquote{coerced weight}), although it is clear that (some aspects of) such structure can be lexically contrastive, as in the case of \posscite{moren01:_distin} \enquote{distinctive weight} or lexical stress, if it is represented as stored prosodic structure. Here, again, the evidence for phonological status comes not necessarily from distinctive status in the lexicon, but rather from the fact that phonological computation is clearly sensitive to the presence or absence of prosodic structure.

Another relevant case is mutually predictable distributions, where some global condition ensures that two dimensions can combine in only two of the logically four possible ways. In a classic contrast\hyp based account only one dimension has to be designated as distinctive and the other must then be treated as redundant, and thus possibly non\hyp phonological (or at least \enquote{derived} in some sense). Classic examples include the issue of vowel and consonant length in (most of) North Germanic \citep[for a recent overview, \cf][]{kristoffersen11:_quant_old_norse_north_german} or the distribution of \ipa{[i]} and \ipa{[ɨ]} following palatalized \emph{resp.\@} non\hyp palatalized consonants in Russian \citep[\egm][]{plapp1996,padgett:_fasl18}, or the connection between vowel length and laryngeal contrast in English pairs such as \alternation{\emph{bead}}{\emph{beat}}; in \cref{sec:foot-intern-struct} I analyse an instance of this phenomenon in Welsh (see also \cref{sec:enqu-vowel-length} for discussion).

A system with Richness of the Base and a non\hyp trivial computation does not require the analyst to choose the one dimension that is distinctive; quite to the contrary, it is incumbent on the grammar to rule out the appearance of disharmonic outputs, which means that the analysis has to take inputs with \enquote{incorrect} distributions into account. Once these are available, it is fully possible that the computation enforces the mutually predictable distribution. It is even possible that multiple combinations of underlying representations and constraint rankings can converge on the correct result, in which case the analysis probably has to be complemented by, say, psycholinguistic studies which can help to identify the correct form of the lexicon in a given speaker.\footnote{It is of course also possible that different speakers may have different grammars, all converging on correct output for the relevant lexical items.} Determining which dimension \enquote{is distinctive} is not possible: all of them may be potentially distinctive, but the outputs are winnowed by the computation. This is the insight which has contributed to the computational turn and the trivialization of representation in OT\hyp based approaches \citep{kirchner97:_contr,flemming05:_deriv}, and, if we accept that the computation is powerful enough, it retains much of its validity: a sufficiently elaborate computation may well take information that is made available to the phonology by the alphabet used to implement lexical contrast and render this information superfluous for contrast purposes. Nevertheless, this redundancy need not, logically, imply that predictable information should be expunged from the phonology, especially if the learner can recover it using evidence from phonological alternations.

A final argument for the existence of predictably distributed yet phonologically distinct symbols comes from the existence of so\hyp called secondary splits. In classic phonemic theory, one mechanism for the appearance of new phonemes was the retention of contrasts after the disappearance of their conditioning environment \citep[\egm][]{twaddell38:_old_high_german,mart55}. However, as \citet{hyman76:_phonol,kiparsky95,janda,bermudez-diachr} and others have pointed out, this approach has no explanation for why the allophony does not disappear when the conditioning context is no longer present. \citet{bermudez-diachr} gives the example of the Indo\hyp Iranian phonologization of postalveolars following the lowering of \ipa{[e]}. In Pre\hyp Indo\hyp Iranian, \ipa{[k]} underwent predictable palatalization to \ipa{[ʧ]} before \ipa{[e~i]}, at which point \ipa{[ʧ]} was an allophone of the phoneme \ipa{/k/}. Once \ipa{[e]} lowered to \ipa{[a]}, however, the existence of the sequence \ipa{[ʧa]} (contrasting with inherited \ipa{[ka]}) clearly established the phonemic status of \ipa{[ʧ]}. What the classic theory is unable to account for is why the lowering of \ipa{[e]} did not lead to the abolition of the allophony of \ipa{/k/}. The answer is that the phonologization of the contrast must have preceded the loss of the conditioning environment. This, in turn, presupposes that phonologically distinct entities may still be in complementary distribution and thus \enquote{redundant} for the purposes of contrast.

In this section I have argued that the relevant sense of \enquote{contrast} in the formulation of the Contrastivist Hypothesis is not unpredictability of surface distribution but rather actual use in lexical representations. It is thus not necessarily the job of synchronic phonology to account for the fact that some of the symbols used in lexical representations are in complementary or near\hyp complementary distribution, or that some of them are used very sparingly, in so\hyp called \enquote{marginal contrasts} \citep[\egm][]{scobbie08:_quasi,hall09}. Note that although this represents a deviation from the classic generative assumption that minimizing redundancy in the lexicon is desirable, inspired by advances in information theory in the middle of the 20th century \citep[\egm][]{cohn10:_labor_phonol}, I do not advocate completely abandoning relatively economical lexical storage in favour of an exemplar theory\hyp style rich lexicon \citep[contrast][]{scobbie08:_quasi}. This is largely because features are emergent in the present model, and thus the learner must extract their existence from ambient data; this means that the existence of a phonological feature must be justified by its participation in unambiguously phonological processes. Since the amount of such robust evidence is usually not very large, speakers will not tend to lexicalize too many distinctions as phonological features.

\subsection{Contrast in stratal OT and redundant features}
\label{sec:contrast-stratal-ot}

A consequence of the architecture laid out in the previous section is the breaking of the link between lexical contrast and the stem level. In classical Lexical Phonology \citep{kiparsky82:_lexic_phonol_morph,kiparsky85:_some_lexic_phonol,kiparsky95}, there is a strict connection between the contrastive status of features and their participation in lexical phonology, formalized via \enquote{Structure Preservation}. However, in the OT model sketched here, it is fully possible for a feature to be redundant, \ie to be disposable for the purposes of establishing lexical contrast, and yet to participate in phonological processes even at the stem level. In OT terms, this follows from the fact that the absence of a contrast can be due not only to the ranking of markedness over faithfulness \citep{bermudez-otero07:_marked}, but also to the ranking of contextual markedness over faithfulness (as in \cref{fig:days-daze-ranking}); in the latter case, faithfulness as such may well dominate markedness, but the effects of contextual markedness constraints mask the contrast that could otherwise have been established. I leave verifying this architectural prediction for future work, as the data I consider in this thesis do not furnish decisive examples.

Another issue regarding the status of the Contrastivist Hypothesis in stratal OT is related to Richness of the Base. As I discussed briefly in \cref{sec:strat-aspects-richn}, Richness of the Base has no problems operating on the stem level, where the absence of restrictions on inputs is quite clear and where the set of features that can and cannot surface can be established with reference to the lexicon (which forms the input to the stem level). At later levels, however, the situation is less clear.

In particular, it is a common assumption in stratal approaches that non\hyp distinctive features become available to the computation postlexically, in seeming violation of the Contrastivist Hypothesis (\cf \citealp{radisic09:_serbian} for some discussion). In stratal OT, this result can be achieved by reranking at later levels, as discussed in detail by \textcite{bermudez-otero07:_marked}. Nevertheless, the status of such redundant features in contrast\hyp based theories remains awkward.

Resolving these issues requires closer analysis of the set of phenomena traditionally catered for by \enquote{postlexical phonology}. It appears highly likely that many cases of putative phonological processes that involve more finely grained phonetic distinctions than those used for lexical contrasts are best treated as not involving any manipulation of phonological symbols, but rather (controlled, language\hyp specific) interface phenomena; for extended discussion of one example, see \cref{sec:recons-surf-undersp} below. Nevertheless, phonological processes crossing word boundaries are also attested (\cf \citealp{scobbieladd} and \cref{sec:except-devo-sandhi} below), which means that any definite answer requires close empirical study.

The possibility of redundant phonological features at shallow strata follows not only from the architecture but also from the study of the life cycle of phonological rules. For instance, a frequent phonetic process is the enhancement of certain contrasts using redundant dimensions \citep{stevens89:_primar_featur_their_enhan_conson,stevens10:_quant,keyser06:_enhan_overl_speec_chain}, and if the process is robust enough, it may undergo the process of stabilization and subsequent phonologization \citep[\cfm][]{bermudez-oterong:_cycles}: the computation, as I have argued in \cref{sec:licens-enhanc-schema}, can readily oblige such a process with the augmentation constraint schema. There is thus nothing except the Contrastivist Hypothesis that prevents us from envisaging the possibility of redundant features introduced by the computation at shallow levels.

Is the Contrastivist Hypothesis thereby falsified? I would suggest that this is not necessarily so. At a stage when some feature is manipulated by shallow phonology but not yet entered the lexicon, it would appear to be in violation of the hypothesis, because it is not used for lexical contrast. However, not being used for contrast can also be seen as an extreme case of being very rare in the lexicon: we could say that the relevant feature is \emph{allowed} to implement contrast, but has not yet done so. Indeed this situation is a natural step along the diachronic path: a feature starts out in the phonetics, continues to shallow phonology, and then starts getting a foothold in the lexicon, going from zero attestations to a few contrasts: for instance, as pointed out by \citet{scobbie08:_quasi}, in some varieties of Scots the distinction between \ipa{[ʌɪ]} and \ipa{[ɑe]}\dash two elements with a distribution that is normally predictable as a result of the Scottish Vowel Length Rule \citep[\cfm][]{scobbie99:_stand_englis_edinb_glasg,mcmahon00:_lexic_phonol_englis}\dash can also be used in a few minimal pairs, as in \emph{gey} \ipa{[ˈɡʌɪ]} `very' and \ipa{[ˈɡɑe]} `guy'. Once again, it is not \emph{actual use} of an element in the lexicon but rather the \emph{possibility} of such use that is relevant for the Contrastivist Hypothesis. In a sense, this is the same conundrum as that discussed above in \cref{sec:probl-empty-segm} in connection with restrictions on inventory structure \citep{blaho-diss,kramer09:_italian}: can the learner posit features that are not used in the lexicon if the evidence from alternations and categorical behaviour is sufficiently robust? It would seem that the answer, in principle, is positive, but that such a situation is perhaps not very stable. This instability pushes predictable phonology further towards lexicalization, driving the life cycle of phonological processes onward \citep{bermudez-diachr,kaisse11:_lexic_phonol}. Empirical testing of this architecture deserves further work (again, the data considered  in this thesis do not seem to be very instructive in this respect), which I must leave for the future.

In this section I have argued that the proper formulation of the Contrastivist Hypothesis must rest not on the existence of minimal contrasts in surface forms but on the dual assumption that phonological computation manipulates phonological symbols and that the set of phonological symbols is precisely the set of symbols allowed (but not necessarily present) in the lexicon. A learner exposed to sufficiently robust evidence consistent with some pattern being phonological (\eg sensitivity to clearly phonological context, sufficient categoricity) may therefore conclude that the elements involved in that alternation are indeed phonological symbols, and make them available to the lexicon. Over time the lexicon will acquire items using previously redundant distinctions, restoring compliance with stronger versions of the Contrastivist Hypothesis. I would suggest that this model represents a much more restricted approach to the problem of redundant feature that previous frameworks stipulating full, often substance\hyp based specification at the postlexical level.

In the next section I will discuss the issue of markedness hierarchies, their relationship with phonetic substance, and the rôle of contrast in this interaction.

\section{Markedness hierarchies and contrast}
\label{sec:mark-hier-contr}

In \cref{sec:constr-compl-struct} I discussed the fact that a geometric theory, such as the version of the PSM presented here, has the property of generating stringent violation sets for markedness constraints similar to those suggested by \citet{delacy2002,lacy04:_marked_optim_theor,delacy2006}. However, the substance\hyp free approach could seem less restrictive than that proposed by \citet[\emph{et passim}]{delacy2006}, because the markedness hierarchies only follow from the structure of representations, while \citeauthor{delacy2006} also connects them very tightly to phonetic substance. In this section I will argue, following \citet{rice07:_marked,rice09:_nuanc,rice11:_what}, that the present approach correctly predicts that markedness\hyp related behaviour of segments \emph{in a given language} follows the predictions of \citeauthor{delacy2006}'s theories of hierarchies, but that the mapping between hierarchy and substance is not part of the universal aspects of phonological computation.

\subsection{Markedness hierarchies}
\label{sec:mark-hier}

\Citet{delacy2006} proposes a theory of markedness that rests on the existence of markedness hierarchies, defined using stringent violation sets. The basic idea is that if a structure $S$ violates some markedness constraint $C$, then all structures that are more marked than $S$ along the relevant dimension also violate $C$, as shown in \ref{delacy-tableau} for place features. The main idea is that a markedness constraint can never single out less marked structures: no ranking of the markedness constraints of the type *[place feature] can make sure that [dorsal] is preferred to [glottal], because surface instances of [dorsal] violate all the same constraints that surface instances of [glottal] do and then some others.

\ex.\label{delacy-tableau}Stringent violation sets according to \citet{delacy2006}\\
\begin{OTtableau}{4}
  \OTtoprow{*\{dors\},\mbox{*\{dors,lab\}},\mbox{*\{dors,lab,cor\}},\mbox{*\{dors,lab,cor,gl\}}}
  \OTcandrow{[ʔ]}{,,,*}
  \OTcandrow{[t]}{,,*,*}
  \OTcandrow{[p]}{,*,*,*}
  \OTcandrow{[k]}{*,*,*,*}
  \OTdashes{1,2,3}
\end{OTtableau}

The existence of these markedness hierarchies has a number of important advantages for deriving markedness\hyp related behaviour, specifically with reference to the phenomena \citet{delacy2006} calls markedness reduction, preservation of the marked, and markedness conflation. I will not review these advantages here, but they are real enough \citep[\cf also][]{causley99:_compl_optim_theor,rice03:_featur}.

As shown in \cref{sec:nonhyp-exha-mark}, the geometric structure of the PSM allows us to derive the stringent violation sets, and thus the OT account of markedness\hyp related behaviour, directly from the subset relations in the structure, à~la \citet{causley99:_compl_optim_theor}. In fact, \citet{delacy2006} derives the stringent violation sets in a similar way, using what he calls the $xo$ Theory of markedness: he assumes that features entering markedness relationships are multivalued, and that violations emerge from subset relations among multiple values. Thus, the hierarchy \dom{[dors]}{\dom{[lab]}{\dom{[cor]}{[gl]}}} is in reality a hierarchy that goes from [$xxx$Place] to [$ooo$Place], where a constraint such as *[$xx$Place] is violated by all [Place] values containing $xx$, \ie\ [$xxo$Place] and [$xxx$Place]. Thus, \posscite{delacy2006} theory is, from a formal perspective, also a structural markedness theory like the one proposed here.

Note, however, that there two important differences between the two approaches. First, a substance\hyp free approach is incompatible with \posscite{delacy2006} second major hypothesis, namely that the mapping between the multivalued $xo$ features and phonetic substance is part of Universal Grammar.\footnote{Although \citet{delacy2006} concedes that at least in one case phonetic substance is phonologically ambiguous, as in the case of place neutralization of nasals to \phonint{ŋ}, which he assumes to be phonologically [glottal] (§2.2.1.1.1).} Second, if markedness equals structural size rather than strings of the $xo$ type, then markedness \enquote{ties} are possible: in $xo$ Theory, the markedness ordering is total, whereas in PSM it is partial. That is, in \posscite{delacy2006} system [labial] is always more marked than [coronal], since no ranking of *[Place] constraints can prefer the latter, whereas in PSM \fea{C-place}{coronal} and \fea{C-place}{labial} are of the same size, and thus their relative markedness has to be determined by the ranking. In other words, in both cases the predictions of \citet{delacy2006} are narrow. Nevertheless, I suggest that his approach is \emph{too} restrictive.

\subsection{Markedness and contrast}
\label{sec:markedness-contrast}

In an OT\hyp based framework of structural markedness, the only logically necessary universals are those that emerge from the interaction of the constraint set \textsc{Con} and the types of structures admitted by the representational system (\cref{sec:enqu-gramm-relev}). For instance, it is possible to derive the fact that bigger (more marked) structures are preferred as triggers of assimilation, or that they resist assimilation more easily. This is due to the process of Preservation of the Marked \citep{rice03:_featur,delacy2006}, which is itself made possible by the fact that faithfulness constraints can protect bigger structures to the exclusion of smaller ones (see \cref{sec:compl-struct-faithf}). Similarly, neutralization due to constraints of the form *[F] will always result in structures of smaller size.

On the other hand, the mapping from structure size to substance is arbitrary in a substance\hyp free approach, so at first blush it would seem that nothing prevents the substance\hyp free theory from generating all sorts of patterns that \citet{delacy2006} argues to be impossible, such as neutralization of place contrasts to [dorsal]. I would suggest, however, that this is not necessarily a bad result \citep[\cfm][]{ramsammyng:_word_spanis}.

As pointed out by \citet{rice09:_nuanc}, generalizations about markedness\hyp related behaviour often only come into their own when there is a contrast to be made. That is, in positions where the phonology allows several elements to appear, markedness relationships tend to exhibit hierarchical structures along the lines identified in markedness research going back at least to \citet{jakobson41:_kinder_aphas_lautg}. However, when the contrasts are neutralized, the outcome of neutralization is much less predictable, with quite a few possibilities attested cross\hyp linguistically.

Consider a situation where, say, consonant place contrasts are neutralized in some position (\eg word\hyp finally). The natural OT account is to assume that this is an instance of markedness reduction, \ie that *[F] constraints ensure that all place features are removed, but that the C-place node remains intact. That is, the phonology will output a bare C-place node in the relevant position. In terms of realization, however, the expression of this C-place node is of course dependent on the system of contrasts in the language: phonologically placeless segments can be glottal in one language, dorsal in another and coronal in a third one, depending on the markedness patterns seen in alternations in the language at large.

When neutralization is avoided, the learner will have additional evidence to set up the markedness hierarchies. As discussed in \cref{sec:non--importance}, these hierarchies will be shaped by extrinsic factors such as diachrony and acquisition, and there the functional tendencies underlying the expression of markedness, à~la \citet{steriade94:_posit,steriade97:_phonet,steriade01}, will make themselves felt. Therefore, in the presence of contrast the effect of substance will be much more pronounced \citep{rice09:_nuanc}, although the ultimate explanation for this fact is not within the purview of the theory of phonology.

The key point here is the arbitrariness of the phonetic expression of unmarked structures, argued for extensively by \citet{rice03:_featur,rice07:_marked,rice09:_nuanc,rice11:_what} but rejected by \citet{delacy2006}, for whom all such differences are to be compelled by hierarchy conflict. Here, I side with \citet{rice11:_what} in assuming that the arbitrariness is indeed greater than prescribed by the narrow predictions of \citet{delacy2006}, but it is clear that more empirical work is needed to resolve this controversy.

In this thesis, I concentrate on in\hyp depth analysis of individual languages rather than on broad cross\hyp linguistic surveys, and the primary contribution of the analysis that follows in \cref{cha:pembrokeshire-welsh,cha:bothoa-breton} is in demonstrating the mechanics of markedness relationships in an OT framework. The main conclusion relevant to the present discussion is that \posscite{delacy2006} generalizations regarding markedness\hyp related behaviour \emph{within a language} are largely correct, but the validity of the tight coupling between structural markedness and substance is more tenuous.

\subsection{Geometry and markedness}
\label{sec:geometry-markedness}

Another potential advantage of the geometrical approach is that it not only derives phonological behaviour related to markedness hierarchies but also offers some solutions to issues in locality.

It has been recognized in the autosegmental literature \citep[\egm][]{avery-rice,piggott92:_variab_featur_depen,odden94:_adjac} that processes such as spreading interact closely with tier structure, and in particular that spreading processes involving some element $x$ which is autosegmentally dominated by $A$ to ignore elements which do not bear $A$; similarly, spreading of $x$ can be blocked depending on the presence of some structure also dominated by $A$. These insights translate naturally into the PSM, \cf in particular \citet{youssef10:_laryn_buchan_scots}.

The version of the PSM proposed in \cref{sec:featural-structure} has the property of expressing notions such as \enquote{contrastive} and \enquote{marked} via feature geometry: segments that are contrastively specified for a feature $A[x]$ are those that bear a (possibly bare) instance of $A$, and \enquote{more marked} segments are characterized by additional structure, making them likely blockers. Segments that are unmarked for a dimension will not bear the node for that dimension.

The same tripartite division of unmarked \vs contrastively specified \vs marked feature values appears in work on vowel harmony by \citet{nevins10:_local}. Working in  a principles\hyp and\hyp parameters framework, he argues that \enquote{Search} processes responsible for harmony may target (or be blocked by) all values of a feature, contrastive ones, or marked ones. This is particularly important for the typology of blocking and transparency, \ie exactly what it usually considered to be the bread and butter of autosegmental theory. Crucially, however, in the present version of the PSM the status of certain structures as contrastively specified or marked is not a diacritic associated with each value by an extrinsic algorithm, but rather emerges from the operation of the contrastive hierarchy.

While in this thesis I do not deal with long\hyp distance processes such as harmony, I would suggest that using something like the present version of the PSM could be a fruitful avenue for expressing \posscite[']{nevins10:_local} insights in an autosegmental, privative theory. If this turns out to be true, then the present approach will have an important advantage over \posscite{delacy2006} $xo$ Theory, since the markedness hierarchies will then emerge from independently supported tier structure. I leave exploring these issues to further research.

\subsection{Partial markedness orders and augmentation}
\label{sec:part-mark-orders}

Another difference of the present proposal vis-à-vis \posscite{delacy2006} is the possibility of partial markedness orders: in the framework espoused here, the relative markedness of two structures of equal size is defined by the constraint ranking rather than representationally. This has some consequences for the treatment of hierarchy conflicts.

\Citet{delacy2006} does not admit markedness reversals: if an element that is more marked along some hierarchy $H$ is preferred to a less marked element in cases of neutralization, this must be due to the existence of some other markedness hierarchy that prevails over $H$ in a particular context. Again, I refer to \citet{delacy2006} for ample discussion of these cases.

In the present theory, apparent markedness reversals have two potential sources. One of them, predicted to be impossible by \citet{delacy2006}, is representational, namely differences in the mapping between phonological structure and substance that go against well\hyp established patterns. I argue in detail for just such a situation in Bothoa Breton in \cref{cha:bothoa-breton}, where, I suggest, voiceless (unaspirated) obstruents are phonologically more marked than voiced ones, contrary to the normal assumption of [$+$voice] as the more marked value in systems not using aspiration \citep[\cfm][]{honeybone05,petrova06:_voice,harris09:_why_final_obstr_devoic_is_weaken}. Another type of neutralization to the more marked value can be driven by augmentation constraints, as hinted at in \cref{sec:licens-enhanc-schema}. This type of neutralization is apparently inevitable in theories based on privative features: if some instances of neutralization involves reduction in structural size \citep[\egm][]{harris97:_licen_inher}, and languages may possess neutralization processes that, depending on the context, may proceed in both directions along a given hierarchy \citep{delacy2006}, then it is inevitable that some neutralization processes should be represented as addition of structure.

Thus, augmentation constraints are merely a formalization of contextual markedness hierarchies that impel neutralization in directions opposite to those required by context\hyp free markedness constraints (note that this is a desired result in view of the fact, discussed in \cref{sec:licens-enhanc-schema}, that augmentation constraints should always mention a context to avoid the Infinite Goodness problem). This means that a total ranking of such augmentation constraints will always define a totally ordered markedness hierarchy, just as required by the $xo$ Theory. This shows an important difference between augmentation constraints and exhaustively interpreted markedness constraints, which can be satisfied by the addition of \emph{some} structure to the element they militate against (\cref{sec:nonhyp-exha-mark}). Augmentation constraints always require the addition of some specific structure, correctly reproducing the effects of \posscite{delacy2006} $xo$ Theory, whereas exhaustive markedness constraints do not express quite the same insights with respect to markedness hierarchies. Therefore, as mentioned in \cref{sec:nonhyp-exha-mark}, I will assume that exhaustive markedness constraints should not be part of \textsc{Con}; for discussion of a complete example, see below \cref{sec:story-ipah}.

Once again, careful comparison of the narrow predictions of \citet{delacy2006} and the more permissive approach to hierarchies espoused here in the spirit of \citet{rice03:_featur,rice07:_marked} lies outside the scope of this thesis. I will now proceed to apply the theoretical framework laid out in the previous chapters to a detailed analysis of the sound patterns of two Brythonic Celtic varieties: the Welsh dialect of Pembrokeshire and the Breton dialect of Bothoa. For more discussion of the conceptual aspects of the theory, buttressed by some specific points of the analysis given in \cref{cha:pembrokeshire-welsh,cha:bothoa-breton}, see below in \cref{cha:disc-altern-analys}.
